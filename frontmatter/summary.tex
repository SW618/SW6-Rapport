\chapter*{Project Summary}
Graphical Interface Resources for Autistic Folk (GIRAF) is a tool developed by students at Aalborg University for their bachelor in Software Engineering.
The development is a collaborative effort of several groups made from these students.
The tool, GIRAF, aims to assist those that have Autism Spectrum Disorders (ASD) by easing them and their caregivers in certain tasks.
This is done by providing a variety of tools through GIRAF.
GIRAF consists of several apps meant to ease some everyday activities by those that have ASD, and help the caregivers facilitate those apps.

For the developers on GIRAF the primary challenge is to successfully cooperate together in the development of a larger system, hence nine groups must work together to improve on the already existing GIRAF.
Each group consists of 3 or 4 students, manage this using Scrum to organise the cooperative development.
As it happens each group have declared that they also use Scrum, as such the project as a whole is organised by a Scrum of Scrums methodology.
The project has been split into four sprints, each with a sprint planning and a sprint end meeting as well as a meeting once a week which works as weekly Scrum for the project.
All groups share a backlog which is maintained by the group who represents Product Owners, this group maintains the primary contact with the customers, alongside the Scrum master group that organise the meetings.
Throughout the report we discuss the process of cooperating on a big team as this, and discuss what changes could be made to the process, as well as discussing which elements are working for the teams.
Alongside this we reflect on our own development process and make changes to it as the project continues.

\bigskip
We had decided that only server management should be an assigned area of work, as such all other groups could freely attempt to solve user stories retaining to any part of GIRAF.
This let to the first sprint mainly being bug and UI fixes that had been revealed at the previous years' user tests, held at the very end of their semester.
We made the Pictosearch library much quicker to use, and made some quality of life improvements to Week Schedule.
At the beginning of the second sprint we decided to focus our efforts.
Realising that spreading out to fix various bugs in all the apps would not get us far, we held a customer meeting and with the information from the customers decided what apps we should focus on.
These apps were Week Schedule, Voice Game, and Launcher which should have an offline mode.
While these apps were the focus, a REST API was also starting development to obtain synchronisation and security.
From our very first meeting with last years developers we had been told that they had tried to implement synchronisation and that it worked, they also mentioned that previous years had tried to do the same, yet never finished.
As it happens their solution did not work and was very insecure, leaving GIRAF with no synchronisation.
Our group worked on the Week Schedule, and Launcher for sprint 2, including improving the scrolling of the Week Schedule application, as well as making offline use possible.

\bigskip
At the beginning of the third sprint our group started to focus on the REST API by producing endpoints.
The goal was not to finish the REST API, but to finish enough endpoints, with a high enough code quality and with guides for next year, such that they could continue developing on the REST API, as it was too big a task to finish in half a semester.
In this report we present the architectural design of the API, as well as the design of each of the developed endpoints and its constituents.

We achieved our goal regarding the endpoints, the REST API contains very structured code and numerous guides have been created for next years developers which should help them quickly set up a build environment and the progress of the REST API, and what should be done next.
As for the overall project goals the multi--project managed to finish the Voice Game and make it standalone, Week Schedule has seen significant progress satisfying the customers needs, however without synchronisation its usefulness is still low, the Launcher allows offline usage.

In conclusion we have made significant headway on the REST API and thus by extension synchronisation for GIRAF while also improving some of the already existing apps.
Our effort for the REST API lets the developers that follow of start off with and already established structure, as well as a few endpoints ready to start client--side implementation.
For Week Schedule, Launcher and Pictosearch our improvements led to more stable apps with more quality of life improvements.
