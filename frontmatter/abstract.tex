We present the process of developing software as a large team of 34 in relation to the app suite GIRAF, a suite aimed at citizens with Autism Spectrum Disorder that aims to ease their life as well as their caregivers.
The project consists of nine group with 3 - 4 students in each group and is organised using Scrum of Scrums.
The system is not started from scratch, an inherited code base exists from 5 previous semesters.
Having no specific area of responsibility for the app suite we explore, improve and develop for several parts of the app suite before focusing on laying the ground work for the development of a REST API.
Despite its lifespan GIRAF has no way to synchronise across devices, we lay the groundwork for this to be a reality, by creating a few of the endpoints required to achieve this.
In this effort we explore the use of several helpful tools and technologies, thus establishing a build environment and a code base following a number of guidelines, to ensure it is organised for the following semester to continue with.