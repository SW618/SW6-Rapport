Developing a complex system is no small task and requires a large team of developers.
We present such development in relation to the app suite GIRAF, a suite aimed at citizens with Autism Spectrum Disorder that aims ease their life and their caregivers.
The project consists of nine group with 3 - 4 students in each group and is organised using Scrum of Scrums.
The system is not started from scratch, an inherited code base exists from 5 previous semesters.
Having no specific area of responsibility for the app suite we explore and improve several parts of the app suite before focusing on laying the ground work for development of a REST API.
Despite its lifespan GIRAF has no way to synchronise across devices, we lay the groundwork for this to be a reality by starting development of a REST API, namely creating a few endpoints.
In this effort we explore the use of several helpful tools and technologies thus establishing a build environment and a code base following a number of guidelines to ensure it is organised for the following semester to continue with.