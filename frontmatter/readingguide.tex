\section*{Reading guide}
Personal pronouns, such as \enquote{we}, refer to the authors of this report. 
All figures are made by the authors unless otherwise specified.
The report is seperated into different parts, firstly some preliminaries, followed by four sprints and an epilogue.
We recommend reading the report in color.

The following words will be used sparingly through out the report:
\begin{description}
    \item[Citizen] \hfill\\
        A citizen is one of the users of the GIRAF application suite, both children and grown ups, with a diagnosis related to the Autism Spectrum.
	\item[Legal Guardian] \hfill\\
        Is a parent or guardian of a citizen.
    \item[Institutional Guardian] \hfill\\
        Is one of the employees working at one of the institution which might use the GIRAF application suite. Will often be refered to as \textbf{Guardian}.
	\item[Customer] \hfill\\
        Aalborg Municipality, is the customer of the GIRAF project.
	\item[Target Group] \hfill\\
        The institutions of Aalborg Municipality, which work with citizens. These represent Aalborg Municipality so their needs are the Customer's needs.
\end{description}
Throughout the report when mentioning citizens or guardians, the pronoun \enquote{he} will be used.
In the report we use \texttt{this monospaced font}, when we talk about specific instances of classes, variables or programming constructs.
We do this to avoid ambiguity, e.g. since \texttt{WeekSchedule}, week schedule and Week Schedule, refers to three different things, namely a class, a concept and an app.
We assume that the reader has experience or knowledge of programming in Java.

In code listings we use \texttt{[...]} to denote omitted code, such that we keep our exampels short and to the point.
