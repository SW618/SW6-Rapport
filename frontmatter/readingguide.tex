\section*{Reading guide}
Personal pronouns, such as \enquote{we}, refer to the authors of this paper. 
All figures are made by the authors unless otherwise specified.
The paper is seperated into different parts, firstly some preliminaries, followed by four sprints and an epilogue.


The following words will be used sparingly through out the paper:
\begin{description}
    \item[Citizen] \hfill\\
        A citizen is one of the users of the GIRAF application suite, both children and grown ups, with a diagnosis related to the Autism Spectrum.
	\item[Legal Guardian] \hfill\\
        Is a parent or guardian of a citizen.
    \item[Institutional Guardian] \hfill\\
        Is one of the employees working at one of the institution which might use the GIRAF application suite. Will often be refered to as \textbf{Guardian}.
	\item[Customer] \hfill\\
        Aalborg Municipality, is the customer of the GIRAF project.
	\item[Target Group] \hfill\\
        The institutions of Aalborg Municipality, which work with citizens. These represent Aalborg Municipality so their needs are the Customer's needs.
\end{description}
Throughout the project when mentioning citizens or guardians, the pronoun \enquote{he} will be used.
