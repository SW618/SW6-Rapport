\newpage
\section{Testing}
For the unit tests we use JUnit.
As a baseline all methods must be tested regardless of their simplicity.
For the Core and Persistence layer we use unit tests and integration tests.
The service layer has no formal testing established as it should not contain anything new, still during development one should test whether or not the JSON response is correct.

It should be noted that the integration tests are in a developer environment, thus there is not actual guarantee it works in other environments.
The integration tests, located exclusively in the persistence layer alongside unit tests, tests whether the DAO works with the model.

The tests are split into 3 segments, firstly resources are added with the \texttt{@Resource} annotation, secondly data is prepared if required using the \texttt{@Before} notation and lastly tests, each of which has the \texttt{@Test} notation to determine it is a test.
For those familiar with the AAA syntax from NUnit \texttt{@Resource} and \texttt{@Before} serves as the arrange part for all tests.
For the Core layer, the unit tests, there will be no \texttt{@Resource} as the Core layer is not testing how a resource works with the rest of the system, but simply methods and objects, as such only instantiating objects is necessary, which is handled with the \texttt{@Before} notation.
For the integration tests \texttt{@Resource} is required for all DAOs tested in a given test class.

\subsection*{Unit/Core test example:}
\begin{lstlisting}
@Before
public void prepareData(){
    department1 = new Department("meme");
    department1.setId(69L);
    department2 = new Department("dank");
    department2.setId(911L);
    user1 = new User(department1, "test1", "hunter2");
    user2 = new User(department2, "test2", "hunter2");
    user1.setId(1L);
    user2.setId(2L);
}

@Test
public void testHasAccessToPublicPictogram(){
    pictogram1 = new Pictogram("rare", AccessLevel.PUBLIC, user1);
    Assert.assertTrue(pictogram1.hasPermission(user1));
    Assert.assertTrue(pictogram1.hasPermission(user2));
}
\end{lstlisting}

\newpage
\subsection*{Integration/Persistence test example:}

\begin{lstlisting}
@Resource
private PictogramDao pictogramDao;

@Resource
private UserDao userDao;

@Resource
private DepartmentDao departmentDao;

@Before
public void getDepartment() {
    department = departmentDao.byName("monstertruck");
}

@Test
public void testPictogramGetAll() throws Exception {
    Collection<Pictogram> pictograms = pictogramDao.getAllPublicPictograms();
    Assert.assertFalse(pictograms.isEmpty());
}

@Test
public void testPictogramGetAllWithUser() throws Exception {
    User u = userDao.byId(department, 1337L);
    Collection<Pictogram> pictograms = pictogramDao.getAll(u);
    Assert.assertFalse(pictograms.isEmpty());
    Assert.assertEquals(pictograms.size(), 4);
}
\end{lstlisting}
