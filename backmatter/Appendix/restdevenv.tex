\section{Setting up the Build Environment}
This is a guide in the basic steps for setting up a developer environment for the GIRAF REST API.
This takes between 5--15 minutes.
\subsection*{Setting up Intellij}
\begin{itemize}
    \item First download and install InteliJ IDEA.
    \item Clone the Git repository for the REST API.
    \item Then start Intellij, and go though the initial start wizzard selecting your preferences.
    \item Then from the main start menu of Intellij select: Import Project.
    \item Select the folder which you've cloned the git repository to.
    \item Select ``Import project from external model''.
    \item Select Gradle.
    \item Press next.
    \item Select default gradle wrapper
    \item Press Finish.
    \item Press ok to import all the gradle project data.
    \item Intellij should now start to its main interface.
    \item Wait for it to stop processing (might take a while the first time on slow computers).
    \item Open the ``Event log'' in the lower right corner, it is the yellow speech bubb
le with en exclamation mark.
    \item In it it will show: ``Frameworks detected: Spring framework is detected in the project Configure".
    \item Press the ``Configure''--link.
    \item You should now see the ``Setup Frameworks'', it should show Spring and two sub-items.
    \item Press OK, with them selected.
    \item Now goto ``File'' -> ``Project Structure'' -> ``Project''.
    \item Under Project SDK select the newest JDK (at least 1.8).
    \item If no SDK are to be found, then press ``New'' then ``JDK'' then navigate to your JDK folder (C:\\Program Files\\Java\\jdk*** on Windows).
    \item Under ``Project language level'' select ``8 - Lambdas, type annotations etc.''.
    \item Press OK to close ``Project Structure''.
    \item Intellij will now Index your files again, this might take a while.
    \item If everything was done sucessfully, then you should be able to Build the REST--API by pressing ``Build'' -> ``Make Project''
\end{itemize}

\subsection*{Setting up Spring}
\begin{itemize}
    \item Now we will configure spring to inject the proper settings.
    \item Navigate to the ``persistence-config.xml'' file in ``persistence/src/main/resources/''.
    \item In the top press ``Change Profiles'' -> then select ``Project'' and ``Default''.
    \item This will configure Spring to use the local in-memory database.
    \item Then you should create a file called ``files.properties'' in ``persistence/src/main/resources/''.
    \item This file should be the same as ``files-prod.properties'' but have paths which fits your machine. (On windows a path such as ``C:/giraf/''
is fine).
    \item This path is used to store the pictogramimages and usericons, which are uploaded though the API.
\end{itemize}

\subsection*{Setting up WildFly}
\begin{itemize}
    \item In order to test the REST--API locally, we will now setup WildFly to run locally.
    \item First download WildFly from \url{http://wildfly.org/downloads/}.
    \item This is made with the ``Java EE7 Full \& Web Distribution'' in mind.gst
    \item Unzip it.
    \item Open the bin folder.
    \item Use the ``add-user'' script appropriate for your platform (.bat for windows, .sh for nix*)
    \item select a) Management User
    \item Enter a username.
    \item Enter a password.
    \item When prompted if you want to add it to any groups, just press enter.
    \item Type ``yes'' to add the user to the realm ``ManagementRealm'', and again for the AS process prompt.
    \item Press enter to close the prompt.
\end{itemize}

\begin{itemize}
    \item Open Intellij again if you closed it.
    \item Open the build configuration menu by pressing the dropdown in the upper right corner and pressing ``Edit Configuration''.
    \item Press the plus sign to add a new configuration.
    \item Select ``JBoss'' -> ``Local'' (Might have to press show 33 more item in order for it to show, look at the bottom of the dropdown).
    \item First press ``Configure''
    \item Then navigate to the install location of your WildFly folder.
    \item Once it is located press ``Register schemas...'', then press OK, and OK again.
    \item Uncheck ``After launch'' (You should use a REST--client here, try out POSTMAN for Google Chrome)\#SELLOUT
    \item Enter your username and password you set before under ``JBoss Server Settings''
    \item If another service is using port 8080 then enter a Port offset i.e. 1. (NVIDIA Network Services does on windows).
    \item It should give you a warning that ``No artifacts marked for deployment'', press the ``Fix'' next to it, select ``dk.aau.giraf.rest :
giraf-rest-1.0.0.ear (exploded)''.
    \item Press OK to exit the configuration menu.
    \item Try to run the WildFly server by pressing run in the upper right corner.
    \item This should build and deploy the WildFly server.
    \item After a while the Output window should show: ``Artifact Gradle : dk.aau.giraf.rest : giraf-rest-1.0.0.ear (exploded): Artifact is deployed
successfully''.
    \item And above it ``Registered web context: /services-1.0.0''.
    \item This means that on the url \url{http://localhost:8080/services-1.0.0} should be the base of the REST--API.
    \item However nothing should be running on the root, try accessing \url{http://localhost:8080/services-1.0.0/pictogram/}.
    \item This should show some JSON for the public pictograms made in the localdata.
    \item Otherwise there should be an error in the Output view for you to debug.
\end{itemize}

\subsection*{Checkstyle}
Optionally you can use the CheckStyle-IDEA plugin to check linting in real-time.
\begin{itemize}
    \item Go to ``File -> settings'', then ``Plugins'' then ``Browse repositories''
    \item Search for ``CheckStyle-IDEA'' and press install.
    \item Press close, then ok.
    \item Restart Intellij
    \item Go to ``File -> settings''
    \item Then ``Other Settings -> Checkstyle''
    \item Under ``Configuration File'' press the plus icon.
    \item Find the ``checkstyle.xml'' file in the root of the Git repository.
    \item Be sure to mark it as ``Active''
    \item Press Ok
\end{itemize}

\bigskip
Enjoy.
