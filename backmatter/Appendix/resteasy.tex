\newpage
\section{RESTeasy}\label{app:resteasy}
\begin{displayquote}
RESTEasy is a JBoss project that provides various frameworks to help you build RESTful Web Services and RESTful Java applications.
It is a fully certified and portable implementation of the JAX-RS 2.0 specification, a JCP specification that provides a Java API for RESTful Web Services over the HTTP protocol.\footnote{\url{http://resteasy.jboss.org/}}
\end{displayquote}

\noindent
The JAX-RS 2.0\footnote{\url{https://jax-rs-spec.java.net/}} standard implemented by RESTeasy, provides a set of annotations which are used to create restful endpoints. These annotations should be placed above the methods which are the different endpoints.

\bigskip\noindent
Example:
\begin{lstlisting}
@GET
@Path("/")
@Produces("application/json")
[...]
public Collection<Sequence> getDepartmentSequences(@Context User user, @QueryParam("title") String title)
    throws NoContentException
{
    [...]
}
\end{lstlisting}

\subsection{Annotations}

\subsubsection{HTTP Request Type}
The first annotation describes the HTTP request used to access the method, and we use one of the following
\begin{itemize}
    \item \texttt{@GET}
    \item \texttt{@POST}
    \item \texttt{@PUT}
    \item \texttt{@DELETE}
\end{itemize}

\subsubsection{Path}
The \texttt{@Path} annotation defines the URL to access the method on. E.g. \texttt{@Path("/department")} will point to \texttt{http://\{giraf-something\}/department}.
The path is appended when a service returns another service, see sequence service in the DepartmentService file for an example.
Moreover parameters can be sent to a method using the path. This is done by writing \texttt{\{var\_name\}}within the path and then \texttt{@PathParam("var\_name") Type name} in the argument list for the method.

\newpage
\noindent
Example:
\begin{lstlisting}
[...]
@Path("/sequence/{id}")
[...]
public Sequence getSequence([...], @PathParam("id") long id) {
    [...]
}
\end{lstlisting}

\subsubsection{Query Params}
In an URL the stuff after the \texttt{?} is key-value pairs of parameters.

E.g. \texttt{http://\{giraf-something\}/department/2/sequence?title=something}, here \texttt{title} is \textit{something}.
These parameters can be retrieved with an annotation similar to the \texttt{@PathParam}annotation.
In the methods arguments, you must write \texttt{@QueryParam("var\_name") Type name}.


\subsubsection{Produce / Consume}
Normally a method in any service in the GIRAF REST API will produce and consume JSON data. This is denoted by using \texttt{@Produces("application/json")} and \texttt{@Consumes("application/json")} annotations above a method.
However some methods such as the ones used to delete objects, should not produce or consume JSON, but just use \texttt{Response} as the return type.
