\section{REST API: Week Schedule Endpoint}\label{sec:weekscheduleendpoint}
This section will describe our progress in developing the service layer of the Week Schedule part of the REST API.
The user story is as follows:
\begin{center}
\userstory{As a developer I would like an endpoint for Week Schedules, such that I can retrieve them from the GIRAF REST API.}
\end{center}

We estimated this user story at 6 EP, unfortunately this was not enough time, and we did not solve all the sub--tasks involved in the user story, as unforeseen problems arose when deleting different objects, the review was late on the two earlier layers, and the report was falling behind.
First we will explain what endpoints we will make, then we will explain how the \texttt{PUT}--request should work for \texttt{Weekschedule}s.
\subsection{Endpoints}
\myref{tbl:weekscheduleservice} contains the endpoints which are involved in the Week Schedule part of the REST API.

\begin{table}[h]
    \footnotesize
    \centering
    \begin{tabular}{lllr}
        HTTP Request    & Path                                      & Method Name           & Status \\
        \midrule
        GET             & \texttt{/}                                & \texttt{getAll}               & Done \\
        GET             & \texttt{/\{wsid\}}                        & \texttt{getById}              & Done \\
        GET             & \texttt{/\{wsid\}/progress}               & \texttt{getAllProgress}       & Done \\
        GET             & \texttt{/\{wsid\}/progress/\{uid\}}       & \texttt{getProgressByUser}    & Done \\
        \tblgrpsep
        POST            & \texttt{/}                                & \texttt{insertWeekSchedule}   & Done \\
        \tblgrpsep
        PUT             & \texttt{/\{wsid\}}                        & \texttt{updateWeekSchedule}   & Done \\
        PUT             & \texttt{/\{wsid\}/progress/\{uid\}}       & N/A                           & Not Done \\
        \tblgrpsep
        DELETE          & \texttt{/\{wsid\}}                        & N/A                           & Not Done \\
    \end{tabular}
    \caption{List of Week Schedule endpoints, they all exist at \texttt{/department/\{id\}/weekschedule}. \{wsid\} indicates the id of a \texttt{WeekSchedule} and \{uid\} refers to the id of a \texttt{User}.}\label{tbl:weekscheduleservice}
\end{table}

Generally for all these are that the DAOs are used to retrieve, update, store and delete information from the persistence layer.
Most important here is the \texttt{WeekScheduleDao}, which were introduced in~\myref[name]{subsec:weekschedulepersistence}.

\subsection{POST--requests}
POST--requests for \texttt{WeekSchedule} are more complex 
Since it is possible to nest builders using RESTEasy and Jackson, we create a \texttt{WeekdayBuilder} to create \texttt{Weekday}s and a \texttt{WeekdayFrameBuilder} to create \texttt{WeekdayFrame}s.
In order to nest the builders properly they should each create the correct object nested from them, i.e. \texttt{WeekdayBuilder} should use \texttt{WeekdayFrameBuilder} to create \texttt{WeekdayFrame}s for it etc.

It is also important that each builder verifies its inputs, e.g. when given the id of a pictogram it should verify that the user which are currently creating the \texttt{WeekSchedule} have access to the pictogram etc.
We also check that the \texttt{WeekSchedule} to be created has a thumbnail which the creator can access, all its users are valid and a title.

\subsection{PUT--requests}
There are two different PUT--requests which can be made, one will update a \texttt{WeekSchedule} and one will update the progress information for a given \texttt{User} on a given \texttt{WeekSchedule}.
The PUT--requests can quickly get complicated, since they have to deal with synchronisation and conflicts, we have previously discussed this when we in sprint 2 allowed for GIRAF to be started offline see \myref{sec:offline_giraf} for more.
In \myref{sec:restws} we describe the \texttt{lastEdit}--variables, we use these to keep track of when the entity was last edited.

We have implemented but not completely tested, the way of synchronisation which goes as follows:
If the local device have had the newest version, and wants to update, then update the data on the server.
If the local device did not have the newest version, and wants to update, then create a copy, and prepend \textit{Conflicted Copy:} to its title.
We previously mentioned this method, which is inspired by Dropbox in \myref{ssec:policy}.

This was supposed to be implemented, such that a conflicted copy would only be created if the two different versions of the \texttt{WeekSchedule}s has changed the same \texttt{WeekDays}, but this is where we stopped development due to time--constraints and as such this will be a future development for next years students.

The PUT--request which was meant to update the progress of a user on a given \texttt{WeekSchedule} have not been implemented due to time constraints.
However the code which is responsible for this behaviour have been thoroughly tested in the unit tests.

\subsection{DELETE--requests}
We have not implemented the DELETE--requests due to time constraints.
There should only be one endpoint for deletion related to week schedules, this is for deleting the week schedule itself.
If a guardian wishes to remove a user from a week schedule, or delete some frames from a day, then it is an update to an existing week schedule, and is done using the PUT--requests mentioned above.