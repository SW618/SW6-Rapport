\section{Documentation}
This section pertains to the \userstory{As a future developer I would like a guide for developing on the REST API, along with a plan for what has been created for the API and what still needs to be implemented, such that I can quickly get an overview and set up my work environment at the start of the project.} and will describe how this is resolved and why the these guides are written.

The REST-API was a rather late addition to GIRAF being introduced this year, as such our semester has been in charge of the design, tools and technologies used in its development.
The groups involved, SW615 and SW618, have followed a rigid set of design and project management guidelines to ensure a certain quality, or at the very least structure in the code.
In order to provide a guide and ease the next years developers into the REST API part of GIRAF and the guidelines that follows, each of the following sections have had a wiki page written, explaining how and when each subject is used in the REST-API, the content of these wiki pages can be found in \myref{app:guides}.
We also added \myref{subsec:general} as a guide because it gives a great overview of what developing on the GIRAF REST API entails.
The plan which the user story requests, is also a guide on the wiki page, but this entails parts of \myref{chp:futureworks} and is therefore not found in the \myref{app:guides}.
Aside from these guides two guides were written by SW615, regarding the Linting tool, and the SQL migrations.

\subsection{Build Environment Setup}
In the start of sprint 3 we spent a substantial time on getting our build environments up and running in order to be able to develop for the REST API.
Since this time could be better spent otherwise, we decided to make a guide which explains the steps needed to get up and running when developing the REST API.
This should substantially decrease the time it takes to get setup, and allow new developers to quickly get up and running.

\subsection{Javadocs}
The Javadocs wiki page explains when and where to use Javadocs, a guideline for this is written specifically for the REST-API as Enunciate also depends on Javadocs.
Aside from Javadocs providing a quicker way to read the purpose, effect and use of code when looking at it as well as providing an overview through the web documentation it provides, for REST-API it helps provide us with even better endpoint documentation through enunciate.
It is important that we ensure the Javadocs is consistent such that Enunciate can efficiently provide client-side developers with the information they require about the REST-API endpoints.
Javadocs provides greater detail for the entirety of the project and is useful for working on the REST-API, however for any instance where a developer needs to work with the REST-API, enunciate creates a far superior way of presenting the required information.

\subsection{Testing}
A wiki page has also been constructed for testing, in order to explain where, when and what type of tests are created as well as how to use the proper notations when testing.
The tests that we use are both unit tests and integration tests.
The obvious reason for having tests is to test correctness, however even more beneficial is the regression testing acquired through having all methods, and the interconnected modules tested together through integration tests.
These integration tests, are the only tests for the ORM created by Hibernate, these tests are beneficial when refactoring to make sure that the changes did not break the any functionality.
This guideline exists to ensure that while GIRAF may not have tests, the REST-API will be tested such that future development can occur without breaking anything.

\subsection{Hibernate Annotations}
We spent alot of time figuring out how to make relationships between classes using Hibernate.
Therefore a guide to show how to create these relations is also created in order to resolve any initial troubles the new developers of GIRAF might run into, something we could have used when we first ourselves started out.
We hope this guide will be something they can use when they continue to expand on the API for other apps in the GIRAF Application Suite.

\subsection{RESTeasy}
As with Hibernate, RESTeasy uses annotations to function.
This guide shows basic use of these annotations in order to implement a service in the REST API.
The guide provides a fundamental understanding, and additional reading should not be needed to start developing.
