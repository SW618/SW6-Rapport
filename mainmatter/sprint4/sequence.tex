\section{Sequence Endpoint --- Service Layer}
This section will describe the continued work on the Sequence Endpoint, more specifically the service layer, which exposes the sequence endpoint to the outside world.
It deals with the same user story as the \myref[name]{sec:sequence}, which is:
\begin{center}
\userstory{As a developer I would like an endpoint for Sequences, such that I can retrieve them from the GIRAF REST API.}
\end{center}

In sprint 3, we completed the two layers Core and Persistence, and now only need to write the services that use these layers to return JSON data.
This service layer uses RESTeasy, like the Pictogram service, described in \myref[name]{ssec:pictogram_service}.
To access any sequence service, a user must go through the department service, hence all services are located at \texttt{/department/\{id\}/sequence}.
The following table (\myref{tbl:sequence_eps}) shows how and where the sequence endpoint can be accessed.

\begin{table}[]
\footnotesize
\centering
\begin{tabular}{rll}
HTTP Request    & Path          & Method Name                   \\
\midrule
GET             &/              & \texttt{getDepartmentSequences}     \\
&\multicolumn{2}{l}{Optional query param: \texttt{title}}\\
GET             &/\{pid\}       & \texttt{getSequence}         \\
\tblgrpsep
POST            &/              & \texttt{createSequence}      \\
\tblgrpsep
PUT             &/\{pid\}       & \texttt{updateSequence}      \\
\tblgrpsep
DELETE          &/\{pid\}	    & \texttt{deleteSequence}	    \\
\end{tabular}
\caption{List of Sequence endpoints, they all exist at \texttt{/department/\{id\}/sequence}.}\label{tbl:sequence_eps}
\end{table}

\subsection{Searching For a Sequence}
By providing the optional query parameter \texttt{title}, the sequences returned can be filtered by title.
The URL used to make such a request will then be: \texttt{/department/\{id\}/sequence?title=\{searchstring\}}.
In \myref{lst:seq_serviceget} the method used for this search is shown.
On line~\ref{lst:seq_methodheader} the query parameter by the name of \texttt{title} is created using the RESTeasy annotation \texttt{@QueryParam("title")}, which enables the query parameter to be extracted from the request URL and used by the method.
In the method body it is first checked if the user making the request is of the given department specified in the URL, and after that on line~\ref{lst:seq_searchif}, it is checked whether or not the optional query parameter was given.
Moreover it is checked if the parameter is not a string containing zero characters, since a search for such a title would be meaningless.
If the \texttt{title} parameter is usable a collection of sequences matching the search query is returned using the sequence DAO.
The method used for this return can be seen in \myref{lst:seqdao_search}.
However if no query parameter or the empty string was given, all sequences available for the department are returned --- again using the sequence DAO.  

\begin{lstlisting}[float, floatplacement=h, caption={The method which returns a list of sequences; and it can be filtered using a query parameter.}, label={lst:seq_serviceget}]
@GET
@Path("/")
@Produces("application/json")
@RolesAllowed(PermissionType.Constants.USER)
public Collection<Sequence> getDepartmentSequences(@Context User user, @QueryParam("title") String title)(*@\label{lst:seq_methodheader}@*)
    throws NoContentException
{
    if (!(user.getDepartment().equals(department)))
        throw new NotFoundException("User is not allowed in the Department");
    if (title != null && title.length() > 0)(*@\label{lst:seq_searchif}@*)
        return sequenceDao.searchByUserAndTitle(user, title);(*@\label{lst:seq_searchdao}@*)
    return sequenceDao.byDepartment(department);
}	
\end{lstlisting}

\subsection{Updating a Sequence}
When updating a sequence using the PUT request.
The body of the request must contain a JSON object which describes the updates that should be made.
Such a JSON object could look as follows:

\begin{lstlisting}[language=json]
{
	"title" : "New Title",
	"thumbnailId" : 2,
	"elementIds" : [ 4, 6, 3, 4 ]
}
\end{lstlisting}

Sending the above JSON object using a PUT request to \texttt{/department/4/sequence/9}, will update the title, thumbnail pictogram, and elements of the sequence with the id 9.
Any property relevant for a sequence is optional in the JSON object, and the order is irrelevant.
As only the properties in the JSON data will affect the update on the sequence.
Updating all properties would require a JSON object like the following:

\begin{lstlisting}[language=json]
{
	"title" : "New Title",
	"ownerId" : 55,
	"accessLevel" : "PRIVATE",
	"thumbnailId" : 2,
	"elementIds" : [ 4, 6, 3, 4 ]
}
\end{lstlisting}

If the goal of the update is to remove all elements from the sequence, the property called \texttt{elementIds} must be an empty array, i.e. \texttt{"elementIds" : []}.

\bigskip

All the methods seen in \myref{tbl:sequence_eps} make up the service layer for \texttt{Sequence}, and as such the user story is completed.