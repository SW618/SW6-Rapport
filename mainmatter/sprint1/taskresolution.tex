\chapter{Task Resolution}
This chapter will describe all the tasks handled in this sprint in varying degrees of detail depending on the significance and uniqueness of the solution.

\section{Update Dependencies}

This sprint contains three tasks which refer to updating dependencies, the three repositories that we are responsible for updating all require us to update the same three dependencies.
\begin{itemize}
    \item localDB from version 5.1.2 - 5.1.5
    \item meta-database from version 3.2.0 - 3.2.3
    \item oasisLib from version 7.2.0 - 9.0.2
\end{itemize}
\kim{I like that you combined these tasks. However the problem is kind of difficult to understand. Maybe try explainin why you want to update the dependencies. Maybe make some kind of dependency diagram or just illustrate the problem. How many components depend on these?. Also please write a line or two about what localDB, oasisLib, meta-database etc. are.}
As can be inferred by the version numbers updating oasisLib is likely to cause errors as it is two non backwards-compatible versions behind.
localDB and meta-database both pertain to the database where meta-database defines all the tables and keys and localDB pertain to querying the database. 
oasisLib contains models and controllers for said models, updating this particular library causes a variety of errors in the repositories sequence and sequence-viewer.\todo[inline]{Der eksistere ingen dokumentation overhovedet om hvad de 3 depencies er, detter er blot hvad jeg har kunne inferere gennem de methods som er i filerne, what do boys? - M}
\kim{Go search some repports for oasisLib until you find an explanation.}

\begin{description}
    \item[Sequence-Viewer] \hfill \\
    For sequence-viewer updating the oasisLib removes a method being used in order to retrieve pictograms as a new controller for pictograms.
    \kim{There is some information missing here. I am not an expert on android development, but how can pictograms be a controller?  }
    Using the method provided by the new controller provides a nullReference error as the new method for the controller does not work the same as the former method, in order to fix this we explicitly retrieve the name of the pictogram as this is no longer part of retrieving an image and provide a nullcheck, something lacking throughout the system considering the amount of nullReference crashes.
    \kim{This sentence is way to long, add more dots. In the first part of the sentence then you use provided and provides, try not to use the same word twice.  }
    \kim{Instead of saying that it work in a different way, then maybe write the signature of the function. Make the analysis of the problem as precise as possible.}
    \kim{I assume that you also update the dependency to use the newest version, you do not seem to mention this when you descrube your solution.}
    \kim{I dont understand the last part of the sentence.}
    \item[Sequence] \hfill \\
    After upgrading to the newest versions of the libraries mentioned above sequence was unable to build. 
    \kim{Sequence with a captial S please :D}
    This was caused by the usage of a deprecated method. 
    The solution to this issue was fixed by changing a single line to use the correct replacement for the deprecated method.
    \kim{Show me what you canged. I crave code ! ;)}
    The line in question is responsible for loading the pictograms from the database into the application. 
    Previously the model had been used directly however in oasisLib 9.0.2 this is unsupported and one should use the \texttt{pictogramHelper} in the \texttt{Helper}--class, for this operation instead.
    After this replacement was made then we informally verified that the application worked identically to the previous version with the old dependencies.
    \kim{This sentence is missing some commas. It is not very precise to say that the model was used direcitly, try to describe it more precicly. }
    \kim{missing dot}
    Then a diff was submitted, approved and landed in the master branch. 
    \kim{Is this process that you are refearing to describe somewhere? If not then you properly should.}
\end{description}
\todo[inline]{Nu har vi to rimelig forskellige beskrivelser af hvordan en task er løst, hvilken en vil vi benytte fremadrette? - M}
\kim{Now you explained the oasisLib problem, what about the other two? }

\subsection{Wiki Migration}
\kim{It is not clear to me why this is a subsection.}
We have the area of responsibility ``Documentation and Wiki'', part of this is ensuring that the information in the Redmine wiki made by the previous GIRAF students is kept since we will depreciate Redmine in favor of Phabricator.
On the Redmine wiki there is a lot of useful information, some of it might be outdated, but most is still useful and should be kept. 

We have taken the task of starting the new wiki on Phabricator, and migrate the useful information from the Redmine wiki to it.
Part of this is to create a structure which the other members of the GIRAF project can use.
It should be noted that the wiki is only used for internal matters inside the GIRAF project.  

It is important that the front page of the wiki is easy to navigate, as this serves as the entry point and from which where all content should be found. 
The contents of the front-page is:

\begin{enumerate}[topsep=0pt,itemsep=-1ex,partopsep=1ex,parsep=1ex]
    \item Actionable Commitments
    \item Guides
    \item GIRAF Project Goals
    \item Useful Links
    \item Sprint Dates
    \item Backlog
    \item Groups and Slack Channels
    \item Wordlist
\end{enumerate}

Most of the content on the wiki are guides which helps the developers with various tasks. 
These should be easy to navigate and have titles which clearly encompass their purpose and content. 
We separate the guides made during this years GIRAF project, which are updated, from the ones made previously. 
Additionally we clearly indicate that the ones constructed previously are not up-to date with the following warning on the top of the page: ``IMPORTANT: This wiki entry has not been updated in 2016''. 

All members of the GIRAF project have access to add to and edit the wiki. 
It is the primary way to share information such as guides and overviews. 
However if this remains unmoderated then the contents of the wiki will most likely become unstructured. 
Therefore in addition the initial migration, we have setup e-mail alerts in Phabricator such that we get a notification every time someone changes the wiki.
Then we will review their additions to ensure that they are located correctly and linked to from the relevant pages etc. 
This will be an ongoing task as part of our area of responsibility.
\section{Consistent File Encoding}
\todo[inline]{TODO}
\section{Responsive Search}
The task was created because the customers expressed concern for the reactiveness of the program, it sometimes felt slow and customers said it felt like nothing was happening.
It is important that responses from a mobile application happen somewhat quickly, as mentioned in~\cite{Roto:2005:NNF:1062745.1062747}, if an application takes more than 4 seconds to load or respond then other feedback than visual should be used.
Therefore it is decided that almost instantaneous feedback is important such that the users feel like the application is working quickly and effective.
This resulted in wanting a change of when calls were made in the PictoSearch application.

The current version of the PictoSearch application searches whenever the uses touches the search button on either the keyboard or on the GUI of the application.
The search button is located in the middle of the screen, to the left of the search field as can be seen on \myref{fig:screenshot_startup}.
\begin{figure}[h]
    \centering
    \includegraphics[width=0.8\textwidth]{figures/img/screenshots/old_startup.png}
    \caption{Screenshot of the initial view presented to the user when launching PictoSearch}\label{fig:screenshot_startup}
\end{figure}

\begin{figure}[h]
    \centering
    \includegraphics[width=0.8\textwidth]{figures/img/screenshots/old_dialog.png}
    \caption{Screenshot of the search spinner shown while seaching in PictoSearch}\label{fig:screenshot_searchspinner}
\end{figure}
This has been changed such that a search happens whenever the text in the searchfield is changed.
Instead of the big slow search spinner which can be seen on \myref{fig:screenshot_searchspinner}, a new smaller text message is instead displayed which tells the user that the application is searching and what is is searching for. 
This search does not remove the software keyboard from the view, neither does it make you unable to type.
This is possible because the search is implemented as an async task, because of this any new search is queued behind the current search query.
This is fixed by calling \texttt{AsyncTask.cancel()}, which cancels the ongoing search call and makes it possible to start a new call immediately.
Because of this anytime a keystroke is made on the keyboard something should happen on the display other than simply filling out the searchfield, be it a text message explaining what is being searched for, or be it the resulting pcitograms returned from the search.
This means that there is instantaneous feedback for the user, which should give the feeling that the application is reacting to what the user is doing, and it does not feel like nothing is happening.

The search button can still be used, and will bring up the old search spinner as usually, but the button has been moved from the middle of the display to the right side of the display as can be seen on \myref{fig:screenshot_newstartup}.
This is a better fit as it resembles other common search engines such as google, and it is also the recommended way to display a search field according to the Niels Norman Group\footnote{https://www.nngroup.com/articles/magnifying-glass-icon/}.
It can also be seen on \myref{screenshot_newUI} that the cancel search button is only shown when there is actually something to remove from the search field, this removes the clutter of information otherwise produced from the old view.

\begin{figure}[h]
    \centering
    \includegraphics[width=0.8\textwidth]{figures/img/screenshots/new_startup.png}
    \caption{Screenshot of the new initial view in PictoSearch}\label{fig:screenshot_newstartup}
\end{figure}

\section{dk.giraf.lib Breaks Gradle Build}
