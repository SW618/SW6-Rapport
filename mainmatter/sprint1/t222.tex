\section{Week Schedule - Unmark scheduele when delete is cancelled}
Week Schedule is the only component of GIRAF which this task affects; and the problem occurs in the following scenario:
When a guardian launches Week Schedule he is first prompted to select the citizen for which he wishes to edit schedules.
A view consisting of the concerned citizen's schedules is then presented to the guardian.
In this view the guardian can among others delete one or more schedules related to the citizen.
This action is performed by tapping the bin in the upper left corner, followed by tapping schedules to mark them for deletion.
When the marking is done the guardian must tap the bin again whereby he is prompted with the choice of either completing or canceling the deletion.
In the current version of Week Schedule the cancellation of a delete does not unmark the marked schedules, hereby presenting outdated and confusing feedback to the user.
The same problem is present in the individual schedules when the guardian wants to delete one or more activities from a given schedule.
This is formulated as the user story: ``In Week Schedule as an administrator, I would like week schedules that I marked while in deletion mode to get unmarked if i cancel the delete.''
This task was added later during the sprint and has therefore not been a part of the planing poker and not received an estimated EP. 

\paragraph{Clearing the data}\hfill\\
The solution to the problem is almost implemented in the current version of Week Schedule; i.e. the functionality needed is present, but not implemented.
In both of the previously described cases where wrongful feedback is given, Week Schedule uses two variables respectively called \texttt{markedSchedules} and \texttt{markedActivities} to store the marking.
The \texttt{markedSchedules} is a \texttt{HashSet} of booleans related to each of the schedules available for selection; and clearing the \texttt{HashSet} is done by simply calling \texttt{.clear()} on the variable.
With the \texttt{markedActivities} variable clearing the marking proved to be more difficult that in the previous case.
Since \texttt{markedActivities} is implemented as an \texttt{ArrayList} with each element being a boolean array, the \texttt{.clear()} did not behave in the same way.
In the \texttt{ArrayList} each boolean array represents a day on the schedule, which means that clearing the entire \texttt{ArrayList} will remove all existing boolean arrays; this will then cause out of bounds exceptions later on if the user tries to make a new selection, because the boolean arrays have zero elements. 
Hence a custom implementation of a clear function is needed.
This is done by filling the \texttt{ArrayList}, \texttt{markedActivities}, with new boolean arrays of appropriate sizes with the default boolean value - false.
This ensures that future markings can be made without crashing the application.
These changes will however not present the user with the appropriate feedback just yet.

\paragraph{Updating the view}\hfill\\
The part of the view in Week Schedule that displays the schedules and activities consists of \enquote{adapters}.
These adapters need to be notified when changes occur to their content, such that the visual representation can be re-rendered.
To notify the adapters in both cases, already implemented functions are used, i.e. \texttt{updateView()} in the schedule selection view, and \texttt{notifyAllAdapters()} in the activity selection view.

\paragraph{Restructuring} \hfill\\
To avoid introducing duplicated code and poor maintainability, the new functionality is placed in functions called \texttt{exitDeleteMode()}.
These functions are called when the user cancels or completes a delete, either by tapping \enquote{cancel}, \enquote{delete} or the back button;
and their purpose is to return to the default mode, i.e.\ not delete mode.
The function used in the activity selection view can be seen in \myref{lst:exitdeletemode}.

\begin{lstlisting}[caption={The \texttt{exitDeleteMode()} function, which returns the application to the default mode}, label={lst:exitdeletemode}]
private void exitDeleteMode(){
    clearMarkingArrayList();
    doingDelete = false;
    //setNewMode(false);
    copyDayButton.setVisibility(View.VISIBLE);
    saveButton.setVisibility(View.VISIBLE);
    scheduleImage.setVisibility(View.VISIBLE);
    setActionBarTitle(getResources()
        .getString(R.string.app_name_week_schedule) + " - " + selectedChild.getName());
    scheduleName.setEnabled(true);
    for(SequenceAdapter a : adapterList){
        a.setDraggability(true);
        a.setMode(true);
    }
    setNewMode(true);
    makeWeekdayButtonsVisible(true);
    notifyAllAdapters();
}
\end{lstlisting}
