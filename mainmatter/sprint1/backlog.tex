At the start of this sprint we attend the Sprint Planning Meeting, where we assign us to tasks, which we afterwards estimate according to the time it takes to complete them.
This chapter explains the process of this, and also presents the tasks we will be working on in sprint 1.

\subsection*{Sprint Backlog}
The following description contains all the tasks that we are working on during the first sprint.
Most of the tasks were distributed at the Sprint Planning Meeting, tasks starting with ``Blocking Task'' are major issues that block other tasks in the sprint backlog and must be resolved, thus they are added to the sprint backlog as soon as they are found.

In order to estimate how much work is required for a task we use a technique known as Planning Poker.
The technique aims to create a consensus through discussion and hidden voting of how long a task will take to complete.
Once all participants have voted the same time estimation one a task, the task has reached an estimation, and the group moves onto the next task.
For the sake of our use, we use half work days as our time unit.
\unsure[inline]{Skal vi forklare hvordan planning poker opnår dette? - Troels Tror måske jeg har gjort det nu ? -Søren Kim? Hvad syntes du om sådan noget som det her? Skal det have paragraph, subsection for sig selv? - Troels}

\begin{description}[style=unboxed]
    \item[Picto Search - It looks like there are no pictograms, until you search for them] \hfill \\ 
        User story: \textit{``As a user, I would like to have some pictograms shown, when I add pictograms to a category.''}\\
        As of now, it looks like there are no pictograms in the Picto Search library upon opening it. 
        A suggestion could be a list of ``latest used'' or ``recommended'' pictograms, just to show some content, indicating that some pictograms already exists.

        This task is estimated at 8 half work days, the primary concern here lies in not only showing something, but implementing it in a meaningful way such as ``Frequently Used'' or similar ideas.
    \item[Picto Search - Responsive Search] \hfill \\
        User story: \textit{``As a User I would like the Picto Search application to feel more responsive when I search for pictograms, such that I don't feel like nothing happens.''} \\
        As of right now the Picto Search tool is slow and one might think it is not working, a search on each keyboard input could be implemented or simply just improve the speed of the search.

        This task is estimated at 8 half work days, aside from changing when the search method is called, this task also pertains to changing the GUI as it makes little sense, and would make even less sense if a responsive search was used.
    \item[General - Use consistent file encoding] \hfill \\
        User story: \textit{``As a Developer I would like all files to use UTF-8 encoding, to insure that no conflicts occur.''} \\
        Using different file encodings (charsets) can lead to issues\footnote{\url{http://tools.android.com/knownissues/encoding}}. 
        Currently many files are in ``us-ascii'' aka. windows-1252, but some files (in the same repository) are of UTF-8, i.e. ``ugeplan'' contains 107 files of ``us-ascii'' and 5 files of ``utf-8''.    

        This task is estimated at 1 half work day, doing this for every repository is bound to give conflicts with other groups working in overlapping repository, as such this task is derived to doing it every time a file has a file-encoding different than UTF-8 is accessed and writing a guide of how to do so, so others can do this for every file they access.
    \item[SequenceViewer - Update dependencies] \hfill \\
        User story: \textit{``As a developer I want all components to use the newest versions of its dependencies.''} \\
        This component is reliant on other components for which it is using outdated versions that may not be backwards compatible.

        This task is estimated at 8 half work days, while all the update dependencies tasks are similar and the update to the \texttt{gradle.build} itself is simple, time is set aside for unforeseen complications it may cause, as well as identifying what has changed, in particular for major patches.
    \item[Sequence - Update dependencies] \hfill \\
        User story: \textit{``As a developer I want all components to use the newest versions of its dependencies.''} \\
        This component is reliant on other components for which it is using outdated versions that may not be backwards compatible.

        This task is estimated at 8 half work days similarly for SequenceViewer.
    \item[Picto Search - Update dependencies] \hfill \\
        User story: \textit{``As a developer I want all components to use the newest versions of its dependencies.''} \\
        This component is reliant on other components for which it is using outdated versions that may not be backwards compatible.

        This task is estimated at 16 half work days, while having the same reasoning as the former update dependencies tasks, for this a complication with retrieving pictograms is already known to be a conflict in the updated version of a dependency, thus it has more time set aside.
    \item[Blocking Task - Custom plugin dk.giraf.lib for gradle breaks build] \hfill \\
        User story: \textit{``As a developer I would like the libraries to be be build-able.''} \\
        Attempting to run the gradle build for Picto Search provides the error, ``Cannot configure the 'publishing' extension after it has been accessed.''. 
        The error has been identified to originate from the dk.giraf.lib and this must be fixed before any library in the GIRAF project can be built using gradle. \todo{Skal vi forklare hvad Gradle er ? Nu når vi bare nævner det, eller er det lige meget?}

        This task is an unforeseen complication found while working on ``Picto Search - Update Dependencies''
\end{description}

\todo[inline]{Andre rapporter har benyttet sig af tabeller osv. for at overskue time estimation og hvor mange opgaver der er osv. Skal vi evt.også gøre det?}
