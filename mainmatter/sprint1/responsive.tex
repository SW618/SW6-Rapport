The task was created because the customers expressed concern for the reactiveness of the program, it sometimes felt slow and customers said it felt like nothing was happening.
It is important that responses from a mobile application happen somewhat quickly, as mentioned in \cite{Roto:2005:NNF:1062745.1062747}, if an application takes more than 4 seconds to load or respond then other feedback than visual should be used.
Therefore it is decided that almost instantaneous feedback is important such that the users feel like the application is working quickly and effective.
This resulted in wanting a change of when calls were made in the PictoSearch application.

The current version of the PictoSearch application searches whenever the uses touches the search button on either the keyboard or on the GUI of the application.
The search button is located in the middle of the screen, to the left of the search field as can be seen on \myref{screenshot_empty}.

This has been changed such that a search happens whenever the text in the searchfield is changed.
Instead of the big slow search spinner which can be seen on \myref{search_spinner}, a new smaller text message is instead displayed which tells the user that the application is searching and what is is searching for. 
This search does not remove the software keyboard from the view, neither does it make you unable to type.
When you do a search for 