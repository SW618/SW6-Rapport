\section{Update Dependencies} 
\kim{This section could use a critical read through. Most of the errors I point out would be found if you went through this section again and did some proofreading. I have not highlighted all the minor ``language'' mistakes, only the major ones.}
In the current state of the Giraf project the individual sub-project, which are applications or libraries for Android, uses a series of common libraries made for the Giraf project. 
Their dependencies are not all updated to use the newest version, this might introduce bugs which have already been fixed. 
For this reason the first order of business, should be updating these such that this year's Giraf project will start with the fewest issues regarding the dependencies. 
For this reason the Product Owner has given these tasks the priority \phigh. 
\kim{You might also want to point out that in addition to updating the dependencies, you will also fix any errors caused by this change. In your current introduction the problem sounds almost trivial, but you write that you plan to use 32 EP to solve the tasks. Which is a bit of a surprise (I dont like to be surprised ;) )}
We have estimated these tasks to take 8 EP each for Sequence and Sequence-Viewer, and 16 EP for Picto Search. 
We have taken three tasks which relate to upgrading the dependencies of the applications: Sequence and the two libraries Sequence-Viewer and Picto Search. 
We took these as they all had the same versions in common to be upgraded, they are: 
\begin{itemize} 
    \item \texttt{localDB} from version 5.1.2 - 5.1.5 
    \item \texttt{meta-database} from version 3.2.0 - 3.2.3 
    \item \texttt{oasisLib} from version 7.2.0 - 9.0.2 
\end{itemize} 
The first two of these \texttt{localDB} and \texttt{meta-database} has only have their patch-number updated, this should indicate a bugfix or small internal corrections, as such they should not introduce any issues in upgrading. \kim{Move this line down below of your explanations of the two libraries.}
\texttt{localDB} is a library which is used to store information in a local database, most of the stored data is received from the remote server when the Giraf Launcher is opened for the first time. 
This is to reduce the inconvenience of having either a slow or no internet connection, however it also introduces some problems regarding keeping the remote and the local database in sync.  \kim{This is a bit confusing and I had to read the sentence several times to understand it. I misunderstood what ``This'' refereed to. If you flip the previous sentence such that the last you say is that localDB is a lib for storing info in a local db, then it is clear that you continue to talk about the local db.}
\texttt{meta-database} is used to create the \texttt{localDB}, the changes therein does not affect any of the applications we are tasked with upgrading. 
  
\texttt{oasisLib} is the connection from the tablet to the remote database. 
In the upgrade from 7.2.0 to 9.0.2, some of the methods used in the applications and libraries to be upgraded were deprecated, and subsequently removed. 
One such method is the one which is responsible for loading pictograms. 
All uses of these methods have to be replaced with the new method of loading pictograms. 
This now uses a helper, \texttt{pictogramHelper}, which has replaced the methods used directly on the model. 
  
\begin{description} 
    \item[Sequence-Viewer] \hfill \\ 
    The Sequence-Viewer used the model directly to get an image, since this method was deprecated when updating \texttt{oasisLib}, it has to be changed. 
    This method was used to replace a pictogram in a view after selecting an option from a multiple choice. 
     To resolve this issue the line shown in \myref{lst:dep-sv-prev} was changed to \myref{lst:dep-sv-upd}. 
     \begin{figure} 
        \begin{lstlisting}[language=java, caption={Sequence-Viewer with deprecated method call. }, label=lst:dep-sv-prev] 
/* [...] */ 
pictogram.setImage( 
    helper.pictogramHelper.getById(id).getImage()); 
/* [...] */ 
        \end{lstlisting} 
    \end{figure} 
    \kim{Please add line numbers to all your code snippets. (It makes it easier at the exam to talk about the code.)}
    \begin{figure} 
        \begin{lstlisting}[language=java, caption={Sequence-Viewer replacement code. }, label=lst:dep-sv-upd] 
/* [...] */ 
// The helper setImage function no longer acquires the pictogram name and causes null exception error 
// This check for null and queries the name by ID to get name 
if(helper.pictogramHelper.getById(id).getName() == null) { 
    pictogram.setName("pictogram_name"); 
} 
else { 
    pictogram.setName( 
        helper.pictogramHelper.getById(id).getName()); 
} 
helper.pictogramHelper.setImage(pictogram, 
     helper.pictogramHelper.getImage( 
        helper.pictogramHelper.getById(id))); 
/* [...] */ 
        \end{lstlisting} 
    \end{figure} 
    Additionally a null-check have been added to assure that a name, locally, is never set to \texttt{null}, as this can cause issues in other methods which assumes that every pictogram has a name. 
     After resolving these issues the application was tested\kim{I dont recall if you wrote how you test stuff, maybe add a small section somewhere and then reference it here.} and it worked in the same way as the previous stable version did. 
     \item[Sequence] \hfill \\ 
    After upgrading sequence to use the newest versions of the libraries mentioned above Sequence was unable to build. 
     This was caused by the use of a deprecated method. 
     The solution to this issue was fixed\kim{Why would you fix the solution, if it works dont fix it.! I assume that you fixed the problem and not the solution, please rephrase.} by changing a single line to use the correct replacement for the deprecated method. 
    The previous code is show in \myref{lst:dep-seq-prev}, and the fixed code is shown in \myref{lst:dep-seq-upd}. 
    \begin{figure} 
        \begin{lstlisting}[language=java, caption={Sequence with deprecated method call. }, label=lst:dep-seq-prev] 
/* [...] */ 
pictogram.setImageBitmap( 
    helper.pictogramHelper.getById(id).getImage()); 
/* [...] */ 
        \end{lstlisting} 
    \end{figure} 
    \begin{figure} 
        \begin{lstlisting}[language=java, caption={Sequence using the replacement code. }, label=lst:dep-seq-upd] 
/* [...] */ 
pictogram.setImageBitmap( 
    helper.pictogramHelper.getImage( 
        helper.pictogramHelper.getById(id))); 
/* [...] */ 
        \end{lstlisting} 
    \end{figure} 
    The line in question is responsible for loading the pictograms from the database into the application. 
    Previously the model had been used directly, however in \texttt{oasisLib} 9.0.2 this is unsupported. 
    The method call \texttt{.getById(id)} returns a pictogram object and on that object the \texttt{.getImage()} method is called. 
    The \texttt{.getImage()} method have been removed from the pictogram model class in \texttt{oasisLib} 9.0.2, which is why the helper should be used instead. 
    After this replacement was made we informally verified that the application worked identically to the previous version with the old dependencies. 
    \item[Picto Search] \hfill \\
    After upgrading Picto Search to use the newest dependencies, no issues were found which was not already present in the old version and therefore already on the task list for the multi-project. 
    Therefore no changes were made to the application itself during the upgrade. 
 \end{description}
 
 After the completion of these tasks, it is now possible to start working on the tasks using these three applications. 
 However we overestimated how much time we would spend on these tasks, we spent at total of 15 EP but had allocated 32 EP for it. 
 This task also gave us a deeper knowledge of how the Giraf project is structured and how Android applications are structured and written internally. 
