\subsection{Wiki Migration}
We have the area of responsibility ``Documentation and Wiki'', part of this is ensuring that the information in the Redmine wiki made by the previous GIRAF students is kept since we will depreciate Redmine in favor of Phabricator.
On the Redmine wiki there is a lot of useful information, some of it might be outdated, but most is still useful and should be kept. 

We have taken the task of starting the new wiki on Phabricator, and migrate the useful information from the Redmine wiki to it.
Part of this is to create a structure which the other members of the GIRAF project can use.
It should be noted that the wiki is only used for internal matters inside the GIRAF project.  

It is important that the front page of the wiki is easy to navigate, as this serves as the entry point and from which where all content should be found. 
The contents of the front-page is:

\begin{enumerate}[topsep=0pt,itemsep=-1ex,partopsep=1ex,parsep=1ex]
    \item Actionable Commitments
    \item Guides
    \item GIRAF Project Goals
    \item Useful Links
    \item Sprint Dates
    \item Backlog
    \item Groups and Slack Channels
    \item Wordlist
\end{enumerate}

Most of the content on the wiki are guides which helps the developers with various tasks. 
These should be easy to navigate and have titles which clearly encompass their purpose and content. 
We separate the guides made during this years GIRAF project, which are updated, from the ones made previously. 
Additionally we clearly indicate that the ones constructed previously are not up-to date with the following warning on the top of the page: ``IMPORTANT: This wiki entry has not been updated in 2016''. 

All members of the GIRAF project have access to add to and edit the wiki. 
It is the primary way to share information such as guides and overviews. 
However if this remains unmoderated then the contents of the wiki will most likely become unstructured. 
Therefore in addition the initial migration, we have setup e-mail alerts in Phabricator such that we get a notification every time someone changes the wiki.
Then we will review their additions to ensure that they are located correctly and linked to from the relevant pages etc. 
This will be an ongoing task as part of our area of responsibility.