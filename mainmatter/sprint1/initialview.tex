\section{It Looks Like There Are No Pictograms, Until You Search For Them}\label{untilSearch}
\userstory{As a user I would like to have some information when I open Picto Search, such that it does not look like there is nothing there.}

The task is prioritised as \phigh~and has been estimated at 8 EP.  
It should be noted that users of Picto Search always are Guardians.
In the current version, the user is presented with no useful information in the initial view as seen in \myref{fig:screenshot_startup}. 
The view is completely empty, there is no indication that searching is an option apart from the search button, but it does not indicate what you can search for nor how to use the search feature.
This is a problem as the users might feel the application is empty, and may be confused as they do not know how to proceed.
When a user launches Picto Search it is to search for pictograms which are to be used in other applications.
Helping the user by hinting at what to do is deemed a good approach in providing the user with information.
This can be done in two ways, without text and with text:
\begin{description}
    \item [Without text] \hfill\\
    For example when Picto Search is opened the search field could already be in focus and the on-screen keyboard could be shown, effectively reducing the number of actions needed in the search workflow and informs the user that they need to type something.
    \item [With text] \hfill\\
    The other approach is showing text to the user telling them what to do, but this approach alone may not be enough as according to Nielsen \cite{nielsen2003usability} a wall of text is deadly for an interactive experience.
\end{description}

Even though the amount of text in the text solution would be insignificant, it still might not work for all users, so an approach combining both solutions is chosen.
Therefore when Picto Search is opened the software keyboard is forced to appear, along with giving focus to the search field; this enables the user to start searching right away without having to tap anything.
By doing this Picto Search is clearly presented to the user as a search tool similarly to the integrated search function of the iOS mobile operating system, which also shows the keyboard and focusses on the search field upon launch\footnote{\url{https://support.apple.com/en-us/HT201285}}.               
A short description of how Picto Search is operated is also shown in the middle of the screen, hereby providing user with information if they are unsure of how to proceed.
The new view can be seen on \myref{fig:screenshot_newstartup}.
The changes to Picto Search described in this section will be evaluated by the users at the sprint review, along side the previously presented modifications from \myref[name]{RSearch}.