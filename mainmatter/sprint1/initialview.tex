\textbf{User Story:} \textit{As a user I would like to have some information when I open pictosearch, such that it does not look like there is nothing there.} \newline

 \kim{It feels like that your analysis is missing somethig. Maybe add something about the users (I guess it is gardians only), is there other applications that they are use to using, maybe some popular apps from the app store, such you can argue that the design you are making will provide the user with an intuative understanding of what to do.}
The task is prioritised as HIGH and has been estimated at 8 EP.
In the current version, the user is presented with no useful information in the initial view as seen in \myref{fig:screenshot_startup}. 
The view is completely empty, there is no indication that searching is an option apart from the search button, but it does not indicate what you can search for nor how to use the search feature.
This is a problem as the users might feel the application is empty, and may be confused as they do not know how to proceed.
When a user launches pictosearch it is in order to search for pictograms which can be used in other applications.
Helping the user by hinting what to do is deemed a good approach in providing the user with information.
This can be done in two ways, without text and with text:
\begin{description}
    \item [Without text]
    For example when Picto Search is opened the search field could already be in focus and the on-screen keyboard could be shown, effectively reducing the number of actions needed in the search workflow and informs the user that they need to type something.
    \item [With text] 
    The other approach is showing text to the user telling them what to do, but this approach alone may not be enough as according to Nielsen \cite{nielsen2003usability} a wall of text is deadly for an interactive experience.
\end{description}

Even though for the text solution the text would be very short, it still might not work for all users, so an approach joining both solutions is deemed the better solution.
Therefore when Picto Search is opened the software keyboard is forced to appear, along with giving focus to the search field; this enables the user to start searching right away without having to tap anything.
A short description of how Picto Search is operated is also shown in the middle of the screen.
The new view can be seen on \myref{fig:screenshot_newstartup}.

\kim{How will you evaluate of this works?}