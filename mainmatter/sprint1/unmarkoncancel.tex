\section{Week Schedule --- Unmark Schedule When Delete is Canceled}
Week Schedule is the only app of GIRAF which this task affects; and the problem occurs in two scenarios:
When a guardian launches Week Schedule they are first prompted to select the citizen for which he wishes to edit schedules.
A view consisting of the concerned citizen's schedules is then presented to the guardian.
In this view the guardian can, among other functions, delete one or more schedules related to the citizen.
This action is performed by tapping the bin in the upper left corner, followed by tapping schedules to mark them for deletion.
When the marking is done the guardian must tap the bin again prompting them with the choice of either completing or canceling the deletion.
In the current version of Week Schedule the cancellation of a delete does not unmark the marked schedules, hereby presenting outdated and confusing feedback to the user.
The same problem is present in the individual schedules when the guardian wants to delete one or more activities from a given schedule, the workflow is the same.
This is formulated as follows:
\begin{center}
\userstory{In Week Schedule as an administrator, I would like week schedules that I marked while in deletion mode to get unmarked if I cancel the delete.}
\end{center}

\subsubsection{Clearing the data}
Most of the prerequisites to solve this issue is already implemented in the current version of Week Schedule for other purposes.
Thus reusing methods which already exist can mostly solve this issue.
In both of the previously described scenarios where wrongful feedback is given, Week Schedule uses two variables respectively called \texttt{markedSchedules} and \texttt{markedActivities} to store the marking.
The \texttt{markedSchedules} is a \texttt{HashSet} of booleans related to each of the schedules available for marking; and clearing the \texttt{HashSet} is done by simply calling \texttt{.clear()} on the \texttt{HashSet}--object.
With the \texttt{markedActivities} variable, clearing the marking proved to be the more difficult scenario.
Since \texttt{markedActivities} is implemented as an \texttt{ArrayList} with each element being a boolean array, the \texttt{.clear()} did not behave in the same way.
In the \texttt{ArrayList} each boolean array represents a day on the schedule, which means that clearing the entire \texttt{ArrayList} will remove all existing boolean arrays; this will then cause out of bounds exceptions later on if the user tries to make a new marking, because the boolean arrays have zero elements.
Hence a custom implementation of a clear method can be used to solve the issue.
This is done by filling the \texttt{ArrayList}, \texttt{markedActivities}, with new boolean arrays of appropriate sizes and with the default boolean value being false.
This ensures that future markings can be made without crashing the application.
These changes will are not enough to present the user with the appropriate feedback just yet as the view must also be updated.
The clearing is handled by the method called on line 2 in \myref{lst:exitdeletemode}.

\subsubsection{Updating the view}
The part of the view in Week Schedule that displays the schedules and activities consists of \enquote{adapters}.
These adapters need to be notified when changes occur to their content, such that the visual representation can be re--rendered.
To notify the adapters we use methods which are already implemented, e.g. \texttt{updateView()} and \texttt{notifyAllAdapters()}.

\subsubsection{Restructuring}
To avoid introducing duplicated code and poor maintainability, the new functionality is placed in a method called \texttt{exitDeleteMode()}, such that instead of having the same code twice as before it is now inside methods rather than copy and pasted when needed.
These methods are called when the user cancels or completes a delete, either by tapping \textit{cancel}, \textit{delete} or the back button;
and their purpose is to return to the default mode, i.e. not delete mode.
The method used for the scenario which shows the week schedule and its activities can be seen in \myref{lst:exitdeletemode}, which as mentioned clears the markings of the arrays on line 2, and the rest of the listing shows code which reset the view to the default mode.
The other scenario did not require changing the visibility of certain buttons as done on line 5-7 in \myref{lst:exitdeletemode} but rather just clearing the array and afterwards asking the Android OS to update the current view as the data of the views has been changed.

\begin{lstlisting}[caption={The \texttt{exitDeleteMode()} method, which returns the application to the default mode.}, label={lst:exitdeletemode}]
private void exitDeleteMode(){
    clearMarkingArrayList();
    doingDelete = false;

    copyDayButton.setVisibility(View.VISIBLE);
    saveButton.setVisibility(View.VISIBLE);
    scheduleImage.setVisibility(View.VISIBLE);
    setActionBarTitle(getResources()
        .getString(R.string.app_name_week_schedule) + " - " + selectedChild.getName());
    scheduleName.setEnabled(true);
    for(SequenceAdapter a : adapterList){
        a.setDraggability(true);
        a.setMode(true);
    }
    setNewMode(true);
    makeWeekdayButtonsVisible(true);
    notifyAllAdapters();
}
\end{lstlisting}
