\subsubsection*{Version Control Policy}
By enabling the system to work offline, conflicting versions become a reality.
Without the system working offline the system quite simply just retrieved data on launch, however with offline possibilities there must be someway to determine whether the system should operate on the data stored locally on the tablet, or whether it should retrieve and replace its data with that stored on the database.
In Ian Sommerville's book Software Engineering \citep{SEBOOK} he speaks of pessimistic and optimistic version control.
Summarising the two; in pessimistic version control resources in use are locked in order to avoid conflicts whereas optimistic version control expects that should conflicts arise while merging, they can be resolved by the user.

\bigskip \noindent
While we implement a pessimistic approach to version control, first let us consider an optimistic approach.

Generally speaking two approaches, each with a variety of specific applications, present themselves in regards to merging; manual and automatic merging.
An automatic solution would require some data point as a deciding factor for what version to use the obvious data being a time stamp.
The issue in doing so is that several devices access the database.
In the scenario where a device is being used in an offline setting, the time between the changes being saved and the device going online again, another device could have updated the database and thus making the database the most recently used.
An attempt at merging could be made, but at some point automatic merging is bound to cause merge conflicts which would require the user to handle the merging resulting in unfamiliar and inconsistent workflow, something that goes directly against what the customer wishes.
Having the merge process being manual might make for a slightly more familiar workflow compared to automatic merging, however the process itself would still be hard to simplify and feel inconsistent compared to any online functionality.
    
On the other hand in pessimistic version control no merging is required as resources are locked.
As it happens locking resources provide its own issue when it is coupled with offline availability as there is no way to announce that a resource must be locked.
With the inability to lock resources a traditional pessimistic approach is out of the question, yet we still want to stay away from manual merging as we believe there is no way to make the process consistent and intuitive.
In order to do this we use a method inspired from how dropbox handles multiple access to the same file; this avoids merging while still allowing a week schedule to be altered simultaneously in an offline and online setting without conflicts as well as keep the workflow consistent.
To keep it simple, when an application is accessed and altered offline, it will simply create a copy of the data to work on such that no conflicts arise.
The copy will receive a name with a prefix that clarifies that it is the offline version such as ``(Off) <scheduleName>''.