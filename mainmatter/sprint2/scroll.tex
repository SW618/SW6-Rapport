\section{Scrolling in Week Schedule}
\userstory{As a user, I would like to be able to have long schedules which are scroll-able, such that I can schedule more in a single day.}

\begin{wrapfigure}{r}{0.45\textwidth}
    \centering
        \includegraphics[width=0.4\textwidth]{figures/img/screenshots/weekplan_schedule.png}
    \caption{An example of a week schedule.}\label{fig:weekschedule}
    \vspace{-20pt}
\end{wrapfigure}

This section presents the resolution of the aforementioned user story, which was given a \phigh--priority by the PO.

First we will explain the reason for this user story, then two solutions will be presented and finally the solution we choose to implement is presented.

\myref{fig:weekschedule} shows a cutout from a week schedule.
Before the changes presented in this section it was not possible to scroll while touching a pictogram.
It was implemented such that if you were logged in as a guardian and dragged while touching a pictogram you would change the order of the pictograms rather than scrolling through the view.
In order to scroll you would have to touch the background in each daily schedule or by touching the scrollbar as seen on \myref{fig:weekschedule}.
The area which allowed scrolling was so small that it made for an awkward use resulting in inadvertently reordering pictograms rather than scrolling and thereby prompting the need for this change.

This resulted in scrolling being difficult since it was hard to avoid accidentally tapping a pictogram, and needed to be changed.
In order to increase the area that responds to scrolling  it was decided to make it possible to scroll while touching a pictogram.
As a result of this decision a new way to reorder pictograms is required for the two features not to be controlled by the same action causing one to be removed.

\bigskip \noindent
At the sprint planning meeting the following tasks were made for this user story:
\begin{eletterate}
    \item Make the entire day's view scrollable
    \item Redesign the reordering mode
\end{eletterate}
The remainder of this section will be divided into two subsections, the first of which will discuss a possible design and the other on resolving the tasks linked to the user story.

\subsection*{To Button or Not to Button} % That is the question
Currently the GIRAF application suite has buttons in the top of the screen which are used for actions such as entering modes for deleting or copying and for saving etc.
One possible solution for dragging is to create yet another button to change the current mode of the week schedule application such that dragging on a pictogram results in dragging a pictogram instead of scrolling.
However, this solution would still be problematic for when you enter the dragging mode and then want to scroll to the pictogram you want to drag.

Another solution is inspired by how many other applications including the standard Android OS and iOS implement dragging of icons.
Longpressing\footnote{An extended tap for a given period set by the OS} a button or an icon in order to start dragging an icon is used in both Android OS and iOS mainly on the home screen to move change the position of different apps.
The same technique can be used in the week schedule, a longpress on a pictogram will cause the application to now drag the pictogram along the movement of the user.
This solution is preferable to the other as users might already associate a longpress with this functionality and therefore we would adhere to the design guideline of recognisable design \cite[p.~51]{DESIGNBOOK}.
As such we deem this the better solution, the next subsection will introduce how this is implemented in the application.

\subsection*{Resolving the Tasks}
The implementation is separated into two parts, each corresponding to one of the tasks mentioned above.
\subsubsection*{Make the entire day's view scrollable}
To resolve this task, firstly we disable the previous way to drag pictograms and the make the entire view scrollable.
Previously a boolean variable called \texttt{isDraggable}, for each pictogram in the week schedule, was set to true.
Changing this to false disabled the ability to reorder pictograms by dragging them, thus freeing the action such that it can be used for scrolling.

The touch information is sent to the touch handler, \texttt{onTouch(MotionEvent)}, for the pictogram, and not the weekday view itself when attempting to drag on-top of a pictogram.
Therefore each child view (i.e. a pictogram) has to send the information of the motion to its parent view (i.e. a weekday).
The information in question is an amount of pixels for the parent view to scroll.
The code calculating this is shown in \myref{lst:actionmove}, and is part of the \texttt{onTouch(MotionEvent)} method, which handles touch input.
Line 1 decides if a scroll should happen or not, it should happen if the user dragged more than two pixels.
In line 3 the distance to scroll is calculated, and in line 5-6 this information is sent to the parent, which in this case is weekday.

\begin{lstlisting}[float, floatplacement=h, caption={The code executed when someone performs a move action.}, label={lst:actionmove}]
if(lastYCoord != -1 && Math.abs(event.getRawY() - lastYCoord) > 2 && !draggable) {
    handler.removeCallbacksAndMessages(null);
    int scrollDistance = (int) (lastYCoord - event.getRawY());
    lastYCoord = event.getRawY();
    ((ScrollView)this.getParent()).scrollTo(0, ((ScrollView)this.getParent())
            .getScrollY() + scrollDistance);
    scrollTime = new Date();

    return true;
}
\end{lstlisting}

\subsubsection*{Redesign the reordering mode}
Next a presentation of how the longpress is implemented.
\myref{lst:longpress} shows the code executed upon long pressing.
It is a function made inside a runnable which means it will be run in its own thread and therefore makes it possible to queue it for execution.
Lines 3-4 makes it so pictograms can be dragged on the screen and lines 5-7 will if a pictogram is being touched lift it up indicating that the pictogram can be dragged, and the tablet will also vibrate briefly so the user knows they can now drag the pictogram.

\begin{lstlisting}[float, floatplacement=h, caption={The longpress function which is queued upon a \texttt{MotionEvent\_Down}, i.e. a touch.}, label={lst:longpress}]
private Runnable mLongPressed = new Runnable() {
    public void run() {
        adapter.setDraggability(true);
        setDraggable(true);
        if(draggingView != null){
            ((PictogramView) draggingView).liftUp();
            vibrator.vibrate(100);
        }
    }
};
\end{lstlisting}

This function is queued when the initial press begins, and if it lasts for more than 500 milliseconds, then it is executed.
If the press is shorter then it is canceled as shown on line 2 in \myref{lst:actionmove}.

\bigskip
In conclusion these fairly small changes result in what we believe to be a more intuitive way to navigate the Week Schedule app based on the assumption that the user is familiar with Android OS and iOS standards; hopefully the customers agree that it is a better solution, which will be disclosed at the customer meeting following sprint 2 end.
