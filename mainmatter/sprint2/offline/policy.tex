\subsection{Version Control Policy}\label{ssec:policy}
In its current state GIRAF does not support synchronisation of data; when GIRAF is initially launched on a device it downloads all data in the remote database, this is the only interaction between the device and the remote database resulting in any changes or additions made being stored locally.
Having no way to synchronise data is a big problem for the use of GIRAF, and since synchronising is something the users want it is also a significant problem.
A way of synchronising data has previously been developed for GIRAF although it is not working yet and is therefore disabled.
This will be further discussed in \myref[name]{sec:current}.

With offline access, a policy for handling conflicts created by work being done while not connected to the internet, is required.
For this discussion we define a conflict as if two persons, A and B, both work on the same version, v1, of a week schedule.
Person A is offline and makes a version, v1a, of the week schedule, while person B is online and creates another version, v2, from version v1 and uploads it to the server.
When Person A regains an internet connection the two versions, v2 and v1a, will be in write conflict with each other.
The implemented solution is inspired from the way Dropbox handles conflicting files \footnote{Information on Dropbox conflict handling can be found here: \url{https://www.dropbox.com/en/help/36}}.

\bigskip
Currently another group of the GIRAF project is working on a RESTful API, which would help synchronise the data on tablets.
The discussion in this subsection will touch upon conflict handling in GIRAF as if the REST API is already implemented; as this is not the case we will also present a temporary solution to be used until the REST API is integrated into the GIRAF apps.

By enabling offline use, synchronisation becomes more complex, as GIRAF will be more conflict--prone and thus require version control to determine whether the system should upload the data that was stored locally while offline, or retrieve and replace its data with the data stored in the database.
In Ian Sommerville's book \textit{Software Engineering}\citep{SEBOOK} he speaks of pessimistic and optimistic version control.
Summarising the two; in pessimistic version control the resources being worked on are locked in order to avoid conflicts whereas optimistic version control expects that should conflicts arise while merging, they can be resolved by the user.
We cannot lock a resource if the user is offline, therefore a pessimistic approach is impossible.
It is also not a good idea to develop a system with inconsistent workflow, such as manual merging, for users which we are told are not skilled with technology; this in turn puts an optimistic approach out of the question.

\bigskip
In order to avoid manual merging we turn to how Dropbox handles conflicts, if a file is updated, the file is overwritten however if there are conflicts, i.e. more than one person has changed the file independently of each other, a copy is created.
A similar approach is wanted for GIRAF, when a conflict is present and the offline tablet regains internet connection, the server will try to automatically merge the two versions, if a merge conflict occurs, i.e. the same week day of the week schedule has been changed the server will instead create a copy of the offline version to be saved, such that there now are two versions of the same week schedule on the server.
A time stamp can be used to check if a conflict has occurred, if the time stamp on the server is the same as the time stamp saved before making changes offline are the same, the offline version can just be uploaded to the server without merging.
The guardian will be notified if a copy is saved, and will then have to figure out what to do, maybe redoing the work on the servers version.
This scenario should rarely happen as the main idea of using the week schedule while offline is for the citizen to be able to see their week schedule, and therefore not editing week schedules.

\bigskip
While this is the solution which is wanted once the RESTful API is launched, currently when an application is accessed and altered offline, it will simply create a copy of the data to work on to simulate the conflict handling as the server is unable to do any of the work in the proposed solution, until it has been launched.
The copy will receive a name with a prefix that clarifies that it is the offline version such as \enquote{Copy <scheduleName>}.
This has no impact on GIRAF in its current state but allows for an easier transition once synchronisation is enabled.
Once the synchronisation features of GIRAF is up and running this functionality should be moved to the server, so that unnecessary copies are not made.

The next subsection will present the changes made to Week Schedule to adhere to the new temporary solution.
