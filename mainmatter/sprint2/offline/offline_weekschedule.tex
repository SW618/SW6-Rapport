\subsection{Changes to Week Schedule}
The first obstacle in making Week Schedule offline capable is to establish some policies about how conflicts between offline uses should be handled.
Because of the goal of the entirety of GIRAF being usable without a connection to the internet, these policies can be adapted to every aspect of GIRAF which involves synchronisation.

\subsection{Version Control Policy}
In its current state GIRAF does not support synchronisation of data, when GIRAF is initially launched on a device is downloads all data in the database, however any changes or additions made are only stored locally.
Having no way to synchronise data is a big problem for the use of GIRAF, and since synchronising is something the users want it is also a significant problem.
A way of synchronising data has previously been developed for GIRAF although it is not working yet.
With offline access a policy for handling conflictsm created by work being done while not connected to the internet, is required. 
The implemented solution is inspired from the way Dropbox handles conflicting files \footnote{Information on Dropbox conflict handling can be found here: https://www.dropbox.com/en/help/36}. 

\bigskip
Currently another group of the GIRAF project is working on a RESTful API, which would help synchronise the databases on tablets.
The discussion in this subsection will discuss conflict handling in GIRAF as if this API was already established, but as it is not it will finally present a present a temporary solution until the RESTful API is launched.

By enabling offline use, synchronisation becomes more complex, as GIRAF will be more conflict-prone and thus requires version control to determine whether the system should upload the data that was stored locally while offline or retrieve and replace its data with the data stored on the database.
In Ian Sommerville's book Software Engineering \citep{SEBOOK} he speaks of pessimistic and optimistic version control.
Summarising the two; in pessimistic version control the resources being worked on are locked in order to avoid conflicts whereas optimistic version control expects that should conflicts arise while merging, they can be resolved by the user.
We cannot lock a resource if the user is offline, therefore a pessimistic approach is impossible.
It is also not a good idea having the users, which we have been told are not that skilled with technology and do not wish for an inconsistent workflow, manually merging any conflicts, which then talks against a optimistic approach.

\bigskip  \noindent
For this discussion a conflict is defined as if two persons, person A and B both work on the same version v1 of a week schedule.
Person A is offline and makes a version v1a of the week schedule, while person B is online and creates a version v2 from version v1 and uploads it to the server.
When Person A regains an internet connection the two versions v2 and v1a will be in write conflict with eachother.
While we implement neither a pessimistic nor optimistic approach to version control, first, let us consider how they would work in GIRAF. 

For optimistic version control merging is required. 
Generally speaking two approaches, each with a variety of specific applications, present themselves in regards to merging; manual and automatic merging.
An automatic solution would require some data point as a deciding factor for what version to use, the obvious data being a time stamp.
The issue in doing so is that several devices access the database, and therefore this might create a conflict.
Alternatively one could save the old timestamp on the device if something is altered when offline and then compare that timestamp to the one on the database once online. 
If they have the same timestamps the offline version should be uploaded to the database, however if they are not the same both the database and the data on the device has been altered and some merging is required. 
An attempt at merging could be made, but at some point automatic merging is bound to require manual labor to fix merge conflicts which would result in unfamiliar and inconsistent workflow, something that goes directly against what the customer wishes.
A manual merge might be required if for instance the two versions have made changed to the same weekday of a schedule, if two versions have not changed the same weekday, merging should be simple and could be made automatic.
Having the merge process being manual every time might make for a slightly more familiar workflow compared to automatic merging where they have to manually merge sometimes, however the process itself would still be hard to simplify and feel unfamiliar to any other functionality. 
Since both automatic and manual merging both become an inconvenience for the user, neither one works and therefore optimistic version control is not a possibility.
 \kim{This section was difficult to understand, I know it is difficult material to convey. I suggest a change of structure to make your arguments more clear.}\todo{Er det blevet bedre nu ? - Søren}

\bigskip
On the other hand in pessimistic version control no merging is required as resources are locked when being changed.
As it happens locking resources provide its own issue when it is coupled with offline availability as there is no way to announce that a resource must be locked.
With the inability to lock resources a pessimistic approach is out of the question, yet we still want to stay away from manual merging as we believe there is no way to make the process consistent and intuitive.


In order to avoid manual merging we turn to how dropbox handles conflicts, if a file is updated, the file is overwritten however if there are conflicts, i.e. more than one person has changed the file independently of each other, a copy is created.
A similair approach is wanted for GIRAF, when a conflict is present and the offline tablet regains internet connection, the server will try to automatically merge the two versions, if a merge conflict occurs, i.e. the same week day of the week schedule has been changed the server will instead create a copy of the offline version to be saved, such that there now are two versions of the same week schedule on the server.
A time stamp can be used to check if a conflict has occurred, if the time stamp on the server is the same as the time stamp saved before making changes offline are the same, the offline version can just be uploaded to the server without merging.
The guardian will be notified if a copy is saved, and will then have to figure out what to do, maybe redoing the work on the servers version.
This scenario should rarely happen as the main idea of using the week schedule while offline is for the citizen to be able to see their week schedule, and therefore not editing week schedules.

\bigskip
While this is the solution which is wanted once the RESTful API is launched, currently when an application is accessed and altered offline, it will simply create a copy of the data to work on to simulate the conflict handling as the server is unable to do any of the work in the proposed solution, until it has been launched.
The copy will receive a name with a prefix that clarifies that it is the offline version such as \enquote{Copy <scheduleName>}.
This has no impact on GIRAF in its current state but allows for an easier transition once synchronisation is enabled.
Once the synchronisation features of GIRAF is up and running this functionality should be moved to the server, so that unnecessary copies are not made.
\kim{I dont see how this solves the problem.} \todo[inline]{Tror det er bedre forklaret nu så man forstår det... Før var det med rest ikke rigtig nævnt, og det med serverens merging var slet ikke med osv.. Jeg synes det giver mening nu - Søren}

The next subsection will present the changed made for week schedule to adhere to the new temporary solution.


\subsubsection{Implementation of Offline Copy}
To implement a mechanism which adheres to the aforementioned policies, we modify the method which is used to save a weekschedule.
In \myref{lst:saveschedule} the changes to \texttt{saveSchedule()} can be seen and consists of an added branch in the if chain from line~\ref{lst:ss_beginchange} to line~\ref{lst:ss_endchange}.
In a scenario where Week Schedule is started in offline mode and the user is trying to save changes to an existing week schedule, the solution is intended to do the following: 
\begin{enumberate}
\item Make a copy of the original week schedule;
\item Prepend \enquote{Copy ---~} to the title of the changed week schedule; and
\item Save the changed week schedule to the local database.  
\end{enumberate}

To make sure that a copy is not saved of the copy created in step 1) a boolean is usedto indicate if the week schedule already is a copy for offline saving such that another copy is not made in step 3).
When online~\ref{lst:ss_flagtrue} this boolean is set to true, thus preventing a new call to \texttt{saveSchedule()} from reaching the branch for offline saving; hereby performing a regular save.
Furthermore the user is notified during the offline saving process that the changed week schedule will be saved as a copy.

\begin{lstlisting}[float, floatplacement=h, caption={Method which is called to save a week schedule, \texttt{[...]} denotes omitted code},label={lst:saveschedule}]
public boolean saveSchedule() {
    /* Setup code and check if title is already used */
    [...] 
    if (isNew) {        
        /* Save as a new week schedule */
        [...]        
    } else if(offlineMode() && !offlineModeCopy) { (*@\label{lst:ss_beginchange}@*)
        //If week schedule is launched in offlinemode and this is not an offline copy
        //do the following:
        actionHelper = new ActionHelper(this);

        //Restore the original name and save a copy of the new name
        String tmpName = scheduleName.getText().toString();
        oldSchedule.setName(oldName);
        //then make a copy of the original schedule
        actionHelper.copySchedule(oldSchedule, selectedChild);
        //Go back to the new name and prepend an indication of it being a copy
        scheduleName.setText(tmpName);
        schedule.setName(getString(R.string.offline_copy) + " - " + scheduleName.getText());
        scheduleName.setText(schedule.getName());
        //Set a flag that indicates that the current schedule is a copy i.e. should be saved
        //and show a notify dialog that informs the user of the offline copy
        //when the dialog is dismissed the saveSchedule() is called again
        offlineModeCopy = true;(*@\label{lst:ss_flagtrue}@*)
        notifyDialog = GirafNotifyDialog.newInstance(
                getString(R.string.dialog_offline_copy_title),
                getString(R.string.dialog_offline_copy_message),
                METHOD_OFFLINE_COPY_ID);
        notifyDialog.show(getSupportFragmentManager(), NOTIFY_DIALOG_TAG);
        return true;(*@\label{lst:ss_endchange}@*)
    } else {
        /* Regular saving i.e. overwrite existing week schedule */
        [...]
    }
} 
\end{lstlisting}
