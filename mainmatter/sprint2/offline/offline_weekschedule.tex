\subsection{Changes to Week Schedule}
Now that the policies of conflict handling have been established we can implement the temporary policy in the week schedule.

\subsubsection{Implementation of Offline Copy}
To implement a mechanism which adheres to the aforementioned policies, we modify the method which is used to save a week schedule.
In \myref{lst:saveschedule} the changes to \texttt{saveSchedule()} can be seen and consists of an added branch in the if chain from line~\ref{lst:ss_beginchange} to line~\ref{lst:ss_endchange}.
In a scenario where Week Schedule is started in offline mode and the user is trying to save changes to an existing week schedule, the solution is intended to do the following:
\begin{enumberate}
\item Make a copy of the original week schedule;
\item Prepend \enquote{Copy ---~} to the title of the changed week schedule; and
\item Save the changed week schedule to the local database.
\end{enumberate}

To make sure that a copy is not saved of the copy created in step 1) a boolean is used to indicate if the week schedule is already a copy for offline saving such that another copy is not made in step 3).
When online~\ref{lst:ss_flagtrue} this boolean is set to true, thus preventing a new call to \texttt{saveSchedule()} from reaching the branch for offline saving; hereby performing a regular save.
Furthermore the user is notified during the offline saving process that the changed week schedule will be saved as a copy.

\begin{lstlisting}[float, floatplacement=h, caption={Method which is called to save a week schedule.},label={lst:saveschedule}]
public boolean saveSchedule() {
    /* Setup code and check if title is already used */
    [...]
    if (isNew) {
        /* Save as a new week schedule */
        [...]
    } else if(offlineMode() && !offlineModeCopy) { (*@\label{lst:ss_beginchange}@*)
        //If week schedule is launched in offlinemode and this is not an offline copy
        //do the following:
        actionHelper = new ActionHelper(this);

        //Restore the original name and save a copy of the new name
        String tmpName = scheduleName.getText().toString();
        oldSchedule.setName(oldName);
        //then make a copy of the original schedule
        actionHelper.copySchedule(oldSchedule, selectedChild);
        //Go back to the new name and prepend an indication of it being a copy
        scheduleName.setText(tmpName);
        schedule.setName(getString(R.string.offline_copy) + " - " + scheduleName.getText());
        scheduleName.setText(schedule.getName());
        //Set a flag that indicates that the current schedule is a copy i.e. should be saved
        //and show a notify dialog that informs the user of the offline copy
        //when the dialog is dismissed the saveSchedule() is called again
        offlineModeCopy = true;(*@\label{lst:ss_flagtrue}@*)
        notifyDialog = GirafNotifyDialog.newInstance(
                getString(R.string.dialog_offline_copy_title),
                getString(R.string.dialog_offline_copy_message),
                METHOD_OFFLINE_COPY_ID);
        notifyDialog.show(getSupportFragmentManager(), NOTIFY_DIALOG_TAG);
        return true;(*@\label{lst:ss_endchange}@*)
    } else {
        /* Regular saving i.e. overwrite existing week schedule */
        [...]
    }
}
\end{lstlisting}
