This section describes how we design and implement \enquote{Offline Mode} capabilities in GIRAF; throughout the rest of the project \textit{offline} and \textit{offline mode} will be defined as \enquote{without internet connection}.
The following user story addresses the problem and was given a \phigh~priority by the PO.

\begin{center}
\userstory{As a guardian I would like to be able to use Week Schedule without internet access, such that I can use it in the woods.}
\end{center}
The week schedule shows the activities of the week, and therefore when an institution is on a trip they need access to their schedules such that they know what activities to do and when to do them.
As of right now the launcher cannot get to the home screen and launch other apps from it without an internet connection.
The first thing to change is making the home screen appear without an internet connection.
Alongside making the launcher work offline, changes to both the Week Schedule and policies on how possible conflicts between server-- and client--side versions should be handled, e.g. a user does some work offline while another does some work online, are required.
With offline use causing more scenarios for which conflicts can arise it is an important issue in making GIRAF available offline.
While the user story only concerns the use of Week Schedule while offline, the Launcher from GIRAF is also affected as explained earlier prompting changes to both components.

The following subsections will elaborate on the aforementioned components of GIRAF especially the changes needed to enable offline use cases.
The first will be the changed made to the Launcher to allow offline usage, followed by the policies for conflict handling, and finally the changes made to the week schedule using the policy.
