This section describes how we design and implement \enquote{Offline Mode} capabilities in GIRAF; throughout the rest of the project \textit{offline} and \textit{offline mode} will be defined as \enquote{without internet connection}.
The following user story addresses the problem and was given a \phigh~priority by the PO.

\begin{center}
\userstory{As a guardian I would like to be able to use Week Schedule without internet access, such that I can use it in the woods.}
\end{center}

We will introduce multiple changes to both the Week Schedule and the Launcher, alongside policies on how possible conflicts between server-- and client--side versions should be handled.
These conflicts arise when a user who is offline makes changes in GIRAF, concurrently with another user.
When the first user then regains internet connection and synchronises with the remote database, the changes made by the second user may be overwritten if a proper policy for conflict handling is not established. \kim{By reading these few lines then it sounds like that offline mode is already a feature and now you need to solve the problem of synchronization, which I guess it not the case. However, even if you fix this small example then I dont think it will have the intended effect, because it will still seem like that you only try to solve one of the problems of synchronization.  I suggest that you either remove the example and then just explain that you are solving the synchronization problem or then you list all the scenarios that you are going to solve, e.g. two client updates the same record, one client update a record that another deletes, two user create a record that get the same ID, etc. The last approach may be useful for you when you need to validate if your solution is sound. Lets talk more about this at our meeting.}
Because of this, conflicts have a pivotal role in a scenario that includes offline use of GIRAF.
While the user story only concerns the use of Week Schedule while offline, the Launcher from GIRAF is also affected prompting changes to both components.

The following subsections will elaborate on the aforementioned two components of GIRAF especially the changes needed to enable offline use cases.

\kim{Based on this introduction I expect sections describing your policy and sections describing the two apps that need to be updated. Looking through the sections if this chapter, I cant see a policy section. Try to either align this introduction with the structure of the rest of the chapter or align the rest of the sections with this section.  }