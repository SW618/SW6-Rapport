\chapter{Introduction}
Autism Spectrum Disorder (\textbf{ASD}) often referred to simply as autism, is a
group of neurodevelopmental disorders that often affects socialisation, communication and learning.
Treatment will often focus on minimising the effects of the disorder in order to improve general quality of life as no cure exists\citep{autism}.
This project is developed in collaboration between sixth semester software students at Aalborg University and institutions located in Aalborg that all specialise in children with ASD.
The goal of the project is to provide these institution with software solutions for the citizen's guardians, both institutional and legal guardians, that can aid them in helping those suffering from ASD.
In the collaboration process institutional guardians from each institution serve as the customer providing general knowledge of the problem domain as well as requirements for the system.

The project, named GIRAF (Graphical Interface Resources for Autistic Folk), has been in development since it was started by Ulrik Nyman, Associate Professor at the department of Computer Science at Aalborg University, in 2011.
Since then it has served as the bachelor project for software students, making this year its sixth iteration of software students.
The goal for the system is to provide a nonprofit and open-source software tool to help citizens.

A total of 34 software students divided into nine groups will proceed with the development of GIRAF.
Each group will have to work together in order to ensure the best outcome.
For all included this provides new challenges that have not been present in previous experience working on smaller projects in independent group environments.
Furthermore the inherited project already contains a large existing code base, unfamiliar to this years' developers providing a further set of challenges.

Because of these changes to the process of previous semesters, much of the project and therefore this paper will discuss the organisation of the multi--project with the other 8 groups on the GIRAF project.
