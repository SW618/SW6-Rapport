\section{Motivation}
When referring to citizens, one must be aware that there exists multiple different types of autism under the umbrella called \textit{the autism spectrum}.
Besides the widely different diagnosis covered by the autism spectrum, the degree or severity of a given disorder greatly affects the nature of any given citizen.
Because of this there rarely is one thing that defines all citizens hence the GIRAF project must accommodate many different use cases, scenarios and configurations.
However, one thing that seems to be consistent across the autism spectrum, is the need for structure and visual guidance.

The citizen's need for scheduling may seem excessive to some, but for citizens it is often necessary for them to handle their everyday activities.
In certain cases the scheduling needs to be upright pedantic in order for some citizens to complete even the simplest of tasks, e.g. taking a bath or using a toilet.
This is what the GIRAF project tries to help the citizens with.

Furthermore some citizens may not be able to cope with large visual objects.
Demands and specifications as these must also be thought into the design of GIRAF, if it aims to provide functionality to a majority of citizens.
For more information on the institutions work with the citizens and the citizen's needs, see the full notes from the presentation \cite{GIRAF20161stMeeting}.

In the following part we present the work done in the initial mini sprint of the project, which we call sprint 0.
This sprint deals with knowledge acquisition, as well as agreeing on development process in the multi-project.

