\section{Autism and GIRAF}
When referring to autists or citizens, as the GIRAF project defines one part of its stakeholders, one must be aware that there exists multiple different types of autism under the umbrella called \textit{the autism spectrum}.
Besides the widely different diagnosis covered by the autism spectrum, the degree or severity of a given disorder greatly affects the nature of any given autist.
Because of this there rarely is one thing that defines all citizens hence the GIRAF project must accommodate many different use cases, scenarios and configurations.
However, one thing that seems to be consistent across the autism spectrum, is the need for structure and visual guidance.
This is represented in GIRAF in the form of schedules, pictograms and work flow consisting mainly of said pictograms.

The need for scheduling may seem excessive to some, but for citizens it is often necessary for them to handle their everyday activities.
In certain cases the scheduling needs to be upright pedantic in order for some citizens to complete even the simplest of task, e.g. taking a bath or using a toilet.

Furthermore some citizens may not be able to cope with large visual objects, e.g. the icons in the GIRAF launcher.
Demands and specifications as these must also be thought into the design of GIRAF, if it aims to provide functionality to a majority of citizens.
