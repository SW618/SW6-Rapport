\subsection*{Gradle}
In order to automate and introduce similar workflow when building the different parts of the Giraf project, a build automation system is integrated into said parts of the Giraf project; \textbf{Gradle}\footnote{\url{http://www.gradle.org/}}.
Gradle is open source and builds upon the strengths and concepts of older build systems, e.g. Apache Ant\footnote{\url{http://ant.apache.org/}} and Apache Maven\footnote{\url{https://maven.apache.org/}}.
Moreover Gradle's syntax for project configurations is based on the JVM scripting language Groovy\footnote{\url{http://groovy-lang.org/}}.
Gradle is intended to be used in large projects and multi-part projects which contains numerous dependencies.
The tool determines which parts of the project already is up-to-date; this ensures that tasks dependent upon said parts does not need to be run again to build the entire project. 

By nature Gradle is plug-in focused, which enables it to be tailored to almost any type of build automation.
The main build configuration file of a project using Gradle is named \texttt{build.gradle}, and it contains information about dependencies, versions of said dependencies, and any tasks that are relevant for the given project.