\subsubsection*{Workflow}
Now that the tools and development method has been presented, we will present how these tools interact with the scrum method we use and also the scrum of scrums.
To do this we will show the workflow on a task from when it is introduced, until the task has been completed and reviewed.
Multiple tasks are ofcourse chosen by each group of the multi-project for each sprint, but it is simpler to explain the workflow using only a single task.

\tikzstyle{box} = [rectangle, draw, fill=green!40, 
    text width=10em, text centered, rounded corners, minimum height=3em]
\begin{tikzpicture}[node distance = 5mm,
    label/.style={align=center, text width=7.5cm},
    arrow/.style={-stealth, line width=1pt}
]
    \node[block, label]
            (task) at (0,0)
            {\textbf{Task is introduced on Phabricator without a priority}};
        
    \node[block, below = of task, label]
            (triage) {\textbf{Product owners of the multi-project give the task a priority}};
            
    \node[block, below = of triage, label]
            (planning) {\textbf{Task is claimed by a project group on phabricator}};
            
    \node[block, below = of planning, label, label]
            (estimate) {\textbf{The task is estimated by the project group}};
            
    \node[block, below = of estimate, label]
            (branch) {\textbf{A branch is created in repository involving the task, which uses the id of the task}};
    
    \node[block, below = of branch, label, label]
            (work) {\textbf{The task is completed and is ready for review, so a diff is created on phabricator}};
            
    \node[block, below = of work, label]
            (review) {\textbf{A reviewer is chosen for the task, and he/she reviews the diff and check if the task is completed.}};

    \node[decision, below = of review, node distance =1cm]
            (ready) {Changes?};

    \node[block, below = of ready, label]
            (land) {\textbf{The diff is applied to the master branch and the task is marked as completed on phabricator}};

    \node[draw=none, right of = ready, node distance = 6cm] (cornerone){};
    \node[draw=none, right of = work, node distance = 6cm] (cornertwo){};
            
    \path[line] (task) -- (triage);
    \path[line] (triage) -- (planning);
    \path[line] (planning) -- (estimate);
    \path[line] (estimate) -- (branch);
    \path[line] (branch) -- (work);
    \path[line] (work) -- (review);
    \path[line] (review) -- (ready);
    \draw[-] (ready) -- (cornerone.center);
    \draw[-] (cornerone.center) -- node {yes} (cornertwo.center);
    \path[line] (cornertwo.center) -- (work);
    \path[line] (ready) -- node {no} (land);
    
\end{tikzpicture}