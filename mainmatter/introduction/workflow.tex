\subsection*{Workflow}
In this section we introduce the workflow for collaboration in the Giraf project, under Scrum of Scrum. 
Groups claim user stories for their sprint backlog at the sprint planning meeting, but it is possible to remove or claim more user stories during the sprint if needed. 
The ability to remove or claim more user stories is a necessity as it is not possible to estimate exactly correct every time although it is preferebly that it does not happen too often such that groups are still generally aware of the tasks that will be completed for a given sprint.
Furthermore by claiming a task outside of the sprint planning meeting there will be no chance to discuss how it could be solved in general, making them harder to estimate and less concrete.
\myref{s1retro} explains how the multiproject reduces the impact of this weakness after the first sprint.

\bigskip 
\noindent
We have chosen to include a code review process in this years Giraf project. 
The reason for this is to increase the quality of the code, and its documentation, such as comments and javadocs. 
This can reduce the speed at which code is introduced, but we have deemed it reasonable due to a lot of the code current is hard to understand, and poorly documented. 
Code submitted to code review is called a diff, since it is the difference introduced by these changes relative to the current version. 
The person submitting code to review is responsible for finding suitable reviewers.
Typically one or more people with experience in the application or people who have done similar changes to other applications are asked to review.
A person can refuse to review a diff if they think they are not suited to review it, or the do not have time. 
A review is over once the one submitting the code for review is confident that what they have made is of a certain quality.
This quality is hard to define, and is subjective.
It should be noted that not all changes to the source code repositories strictly must go though the code review process.
Small insignificant changes, such as correcting spelling, or small graphical bug fixes can be pushed directly. 

In \myref{fig:workflow} we have constructed a flow diagram for the conceptual workflow for the Giraf multi-project. 
\kim{The workflow figur is not really used in this section. I think it could be an advantage to move it(the reference) to the start of the section, then present the workflow. Afterwards you can discuss pros and cons of the workflow.}

\begin{figure}[H]
	\centering
	\begin{tikzpicture}[node distance = 0.5cm, auto]
	    \footnotesize
	    \node[wideblock]
	            (task) at (0,0)
	            {User story is introduced on Phabricator without a priority, either by a developer or the customer};
	        
	    \node[wideblock, below = of task]
	            (triage) {The PO of the multi-project give the user story a priority or discard it};
	            
	    \node[wideblock, below = of triage]
	            (planning) {User story is claimed by a project group on Phabricator};
	            
	    \node[wideblock, below = of planning]
	            (estimate) {The User Story is estimated by the project group};
	            
	    \node[wideblock, below = of estimate]
	            (branch) {A branch is created in git repository involving the user story, which uses the id of the user story};
	    
	    \node[wideblock, below = of branch]
	            (work) {The user story is completed and is ready for review, so a diff is created on Phabricator};
	            
	    \node[wideblock, below = of work]
	            (review) {A reviewer is chosen for the user story, and he/she reviews the diff and check if the user story is completed.};

	    \node[decision, below = of review]
	            (ready) {Changes?};

	    \node[wideblock, below = of ready]
	            (land) {The diff is applied to the master branch and the task is marked as completed on Phabricator};
	            
	    \path[line] (task) -- (triage);
	    \path[line] (triage) -- (planning);
	    \path[line] (planning) -- (estimate);
	    \path[line] (estimate) -- (branch);
	    \path[line] (branch) -- (work);
	    \path[line] (work) -- (review);
	    \path[line] (review) -- (ready);
	    \node[draw=none, right of = ready, node distance = 6.5cm] (cornerthree){};
	    \draw[-] (ready) -- node {yes} (cornerthree.center);
	    \path[line] (cornerthree.center) |- (work);
	    \path[line] (ready) -- node {no} (land);
	    
	\end{tikzpicture}
	\caption{Simplified workflow for the Giraf multi-project.}
	\label{fig:workflow}
\end{figure}
