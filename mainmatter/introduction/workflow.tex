\subsection*{Workflow}
Now that the tools and development method has been presented, we will present how these tools interact with the Scrum method used and also the Scrum of Scrums. \kim{Sounds confusion, please rephrase. Specifically I am unhappy with the ``used and also'' formulation.}
To do this we will show the workflow on a task from\kim{...of a task, starting from...} when it is introduced\kim{Are task really introduced? try to be more precise.} till\kim{-l} the task is considered completed through use of a flow diagram.
\kim{I just realized that you never defined what a task is. I am guessing that a user story contains several task or how does it work? However, you say that a backlog contains tasks as oppose to user stories. Please explain.}
Multiple tasks are chosen by each group of the multi-project for each sprint, but it is simpler to explain the workflow using only a single task. \kim{you alraedy said that on the line above.}
\kim{Try to add terms such as product backlog and sprint backlog.}
The aforementioned flow diagram is \myref{fig:workflow}. \kim{Aforementioned is properly one of my favorite words. It is AWESOME.}

\begin{figure}[H]
	\centering
	\begin{tikzpicture}[node distance = 0.5cm, auto]
	    \footnotesize
	    \node[wideblock]
	            (task) at (0,0)
	            {Task is introduced on Phabricator without a priority};
	        
	    \node[wideblock, below = of task]
	            (triage) {Product owners of the multi-project give the task a priority};
	            
	    \node[wideblock, below = of triage]
	            (planning) {Task is claimed by a project group on Phabricator};
	            
	    \node[wideblock, below = of planning]
	            (estimate) {The task is estimated by the project group};
	            
	    \node[wideblock, below = of estimate]
	            (branch) {A branch is created in repository involving the task, which uses the id of the task};
	    
	    \node[wideblock, below = of branch]
	            (work) {The task is completed and is ready for review, so a diff is created on Phabricator};
	            
	    \node[wideblock, below = of work]
	            (review) {A reviewer is chosen for the task, and he/she reviews the diff and check if the task is completed.};

	    \node[decision, below = of review]
	            (ready) {Changes?};


	    \node[wideblock, below = of ready]
	            (land) {The diff is applied to the master branch and the task is marked as completed on Phabricator};
	            
	    \path[line] (task) -- (triage);
	    \path[line] (triage) -- (planning);
	    \path[line] (planning) -- (estimate);
	    \path[line] (estimate) -- (branch);
	    \path[line] (branch) -- (work);
	    \path[line] (work) -- (review);
	    \path[line] (review) -- (ready);
	    \node[draw=none, right of = ready, node distance = 6.5cm] (cornerthree){};
	    \draw[-] (ready) -- node {yes} (cornerthree.center);
	    \path[line] (cornerthree.center) |- (work);
	    \path[line] (ready) -- node {no} (land);
	    
	\end{tikzpicture}
	\caption{Simplified workflow.}
	\label{fig:workflow}
\end{figure}
\kim{I like that you use tikz, being able to use tikz is a good skill to have.}
\kim{Where do tasks come from? I assume that they either originates from customers or developers. Actually I would assume that customers created user stories (or the product owners created them on their behalf) as oppose to tasks. }
\kim{What is this repository that you are talking about. This is the first time you mention a repository.}
\kim{I know that it is out of the scope of this workflow but how do you know how many tasks a group should put on their sprint backlog? }
\kim{How is a reviewer chosen?}
\kim{These are some of the questions I had after looking at the workflow diagram. I am not saying that you have to explain every little detail, you need to decide what is important and then make sure to explain that.}
\kim{Now that you have explained the workflow then I would like to know why you designed it this way. What pros and cons does it have? }
