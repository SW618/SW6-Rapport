\subsubsection*{Development and Organisational Tools}
Previously the project have used the project management tool Redmine.
However an opinion that is mostly consistent throughout all the papers from the previous semester is that Redmine should be replaced as it, according to last years students, is not a good tool.
Redmine offers tools for communication, calendars, documentation, issue tracking, a Wiki and more.
The primary issue with Redmine according to last years students refer to its lackluster communication solution.
As this is a problem stated throughout a majority of the papers the web application Slack is used as the communication tool between the groups involved in development.
Furthermore it was decided at the initial meeting to replace Redmine, with another software development tool: Phabricator, an open source, software development and project management platform. 
As Phabricator is still an unfamiliar tool issues may arise, however from the initial overview it seems to be geared towards agile development with applications specific to user stories and backlog tracking.
The workflow established for Phabricator is also very geared towards code review and documentation, something the groups agree seems very lackluster and should have more focus this semester\footnote{For more information about Phabricator visit their website at \url{http://phabricator.org/}}.
Lastly with Scrum meetings happening on a weekly basis a shared Google Docs folder contains all agendas and summaries from meetings.
Furthermore, while there is a primary rapporteur for each meeting, Google Docs allows everyone to pitch in for both the agenda and the summary which in turn means that if someone is not satisfied with the information in the summary, they can also add more information in real-time.

\subsection*{Gradle}\label{subsec:gradle}
In order to automate and introduce similar workflow when building the different parts of the GIRAF project, a build automation system is integrated into all parts of the GIRAF project --- \textbf{Gradle}\footnote{\url{http://www.gradle.org/}}.
Gradle is open source and builds upon the strengths and concepts of older build systems, e.g. Apache Ant\footnote{\url{http://ant.apache.org/}} and Apache Maven\footnote{\url{https://maven.apache.org/}}.
Moreover Gradle's syntax for project configurations is based on the JVM scripting language Groovy\footnote{\url{http://groovy-lang.org/}}.
Gradle is intended to be used in large projects and multi-part projects which contains numerous dependencies.
The tool determines which parts of the project already is up-to-date; this ensures that tasks dependent upon said parts does not need to be run again to build the entire project. 

By nature Gradle is plug-in focused, which enables it to be tailored to almost any type of build automation.
The main build configuration file of a project using Gradle is named \texttt{build.gradle}, and it contains information about dependencies, versions of said dependencies, and any tasks that are relevant for the given project.
                                       
Alternatives to using Gradle as build automation system are for example the aforementioned Maven and Ant tools, however these tools are dependent on writing build scripts in XML which is a markup language not suited for implementing logic into the automation process.
Furthermore Google recommends using Gradle and provides resources such as guides\footnote{\url{http://tools.android.com/tech-docs/new-build-system/user-guide}} and plug-ins for automating Android application builds using Gradle.
