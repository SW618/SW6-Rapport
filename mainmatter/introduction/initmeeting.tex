\section{Development Method and Organisational Tools}
The entire GIRAF project met to discuss the information gathered from last years reports, to discuss pros and cons of the process they used.
Last year's reports recommended to change different aspects of their development process, and their advice has been taken into consideration.
This section will first present the chosen development method as has been chosen at this meeting, and will finally present the tools that will accommodate the development method.


\subsection*{Development Method of the Multi-Project}
The previous semester of the GIRAF project used the agile development method \textit{scrum of scrum}, which is a scaled version of \textit{scrum}.
This is also the choice for the multi-project of 2016, but where as last year they had three-layers of scrum, as the multi-project was separated into three further sub-projects:
\begin{itemize}
	\item Applications
	\item Database
	\item Build \& Deployment
\end{itemize}

This is changed for the multi-project of 2016, as many of the reports from last year recommended this change, such that any group would be able to work on any tasks which occur during the project.

In order to explain how scrum of scrum works, a brief explanation of scrum is now presented.
This is in turn also how we will be working internally in group SW618F16. 


\subsubsection*{Scrum}

We use parts of the version of scrum as defined by Sommerville in his book, Software Engineering in chapter three \cite{SEBOOK}.
The parts used are presented here.

\begin{description}
	\item[Daily scrum] \hfill \\
	Everyday we start out by having a daily scrum, which is a meeting where each person in the group is asked the following questions:
		\begin{itemize}
		    \item What did you do yesterday? 
			\item What will you do today?
			\item What obstacles are in your way or slowing you down?		
		\end{itemize}
		This helps make sure that all group members know what each other are doing, while also giving an opportunity to ask for help if needed.

	\item[Sprints] \hfill \\
	A sprint is a working period where the group sets out to complete different tasks before the sprint end. 
	\item[Product Backlog] \hfill \\
	A product backlog is a list of tasks which are to be done in order for the project to be complete. 
	It replaces traditional requirements lists, and can develop throughout development, more tasks can be added throughout the project.
	When a sprint starts, some of these tasks are chosen by the group, and these tasks are then to be estimated in half-days, and completed before the sprint end.
	All tasks on the backlog are written in the form of user stories like this:``As a <type of user>, I want <some goal> so that <some reason>''.
	\item[Sprint Backlog] \hfill \\
	A sprint backlog contains the tasks chosen from the product backlog by the development team to be completed in the sprint.
	As these tasks are chosen, it is not allowed to add more tasks during this period.
	If it is discovered after estimating the tasks, that there is simply too much work to be done in the sprint, it is okay to contact the scrum master to reduce the workload.
	\item[Scrum Master] \hfill \\
	The scrum master is a role of one of the developers of the project, it is the scrum masters job to make sure that the development team uses scrum correctly, and also helps the product owner organising the product back-log.
	\item[Product Owner] \hfill \\
	The product owner is another role, which deals with understanding what the customer wants.
	The product owner is also in charge of the product backlog, and makes sure to prioritize the tasks here, and makes sure that the tasks are understandable such that it is easy for a developer to proceed with the development of the task.
	\item[Scrum Board] \hfill \\
	A scrum board is a board which contains all tasks from the sprint-backlog.
	These tasks can have different sub-tasks, and these are then moved from different columns depending on where they are currently recide in development.
	They can be in four different columns:
	\begin{itemize}
		\item To-do
		\item Doing
		\item Review
		\item Done
	\end{itemize}
	The scrum board works well to make sure that progress is visible, and to show what different tasks are currently being worked on.
	\item[Burn down Chart] \hfill \\
	The burn down chart makes gives estimates on the progress of a sprint, by calculating the total number of half-days needed to complete the tasks on the sprint-backlog, and as such requires that all tasks in the sprint backlog are time estimated.
	At every daily scrum all time estimates of the completed tasks are summed, and a line is drawn on the chart showing the progress of the tasks.
	The chart has a tendency line which shows how progress should be coming along, and groups can then track if they are ahead or behind according to their estimated development times of their tasks.
\end{description}

All these tools define the scrum method we use in our group, they are not independent of the scrum of scrum which is used in the multi-project, which will be explained next.

The scrum of scrum consists of multiple groups, all using scrum, and then having ambassadors from every group meet and do the scrum meetings \cite{SCRUMBOOK}.
The structure of the scrum of scrum can be seen on \myref{fig:scrumofscrum}

\begin{figure}
\centering
\includegraphics[scale=0.4]{figures/scrumofscrum.png}
\caption{The structure of scrum of scrum. Figure from \cite{scrumofscrumfigure}.}
\label{fig:scrumofscrum}
\end{figure}

Every group will perform a daily scrum individually, and the multi-group will perform a scrum meeting answering the same questions as one would at daily scrum but instead it is just the ambassadors from each group at this scrum meeting, whom then speak on the group's behalf.
According to MountainGoatSoftware it is not necessary to have every day, and many organisations have them two to three times a week instead.\footnote{https://www.mountaingoatsoftware.com/agile/scrum/team}
For the GIRAF project 2016, we have chosen to have one weekly scrum every Wednesday instead.
The reasoning for this is that, the reports last year complained about having too many meetings, and while many organisations might have two to three a week, these organisations do not have courses to attend to, which also require a lot of time.
As the scrum of scrum meetings are held weekly the questions asked at these meeting change to asking: ``...since last we met?'' etc.
There are 4 other meetings than the weekly scrum of scrums specified for the multi-project:

\begin{description}
	\item[Sprint Planning] \hfill \\
	This meeting is held at the start of a sprint.
	The sprint planning meeting is where the tasks from the product backlog will be chosen by all groups to put on their own sprint backlog.
	This means that the multi-project has a single product backlog for the multi-project, which is where all tasks come from.
	\item[Sprint Retrospective] \hfill \\
	The sprint retrospective has the job of reviewing and improving the development process. 
	This is where any issues with the process is brought up in order to improve upon the process of the project.
	\item[Sprint Review] \hfill \\
	The customers attend the sprint review meeting so we can show them the progress of the applications, to gain valuable feedback, to create new tasks, and to help prioritize the tasks. 
	\item[Presentations] \hfill \\
	Sometimes a group has something important to present to the rest of the groups, and therefore a presentation will be made to inform the other groups of the news.
\end{description}

Each sprint will be approximately 3 weeks, which gives time for approximately 4 sprints throughout the project.
At the start of each sprint the sprint planning meeting is held, afterwards the groups will estimate their chosen tasks.
When the sprint is over the sprint retrospective it held, and finally the sprint review.
The last two meetings might interchange, depending on the availability of the customers.
It has to be mentioned that it is not a requirement for other groups of the multi-project to follow the scrum development method.

The scrum master of the multi-project will be group SW613F16, and they are the ones to make sure the other groups of the project adhere to the specified process.
The scrum master is one of many areas of responsibility given to the groups this semester, which will be further explained in the next section.


\subsubsection*{Areas of Responsibility}
Every group of this years multi-project have an area of responsibility.
These responsibilities differ much in terms of amount of work, yet they are all important.
The different areas of responsibility are as follows:

\begin{description}
	\item[Server and Database] \hfill \\
	This area consists of maintaining the server and the database, including migrations and other maintenance and support tasks. They are essentially an internal IT service for the rest of the project. This area is covered by two groups: SW611F16, and SW616F16.
	\item[Social and Google Analytics] \hfill \\
	It is important that the groups of the multi-project know each other, and that they are all friendly with each other. 
	This will make it easier for people to ask for help and work together. 
	Google Analytics is a tool which help analyze and keep statistics of many things, such as how customer's find your application, or sending bug reports to the developers.
	The groups responsible for this is group SW612F16
	\item[Scrum Master] \hfill \\
	As mentioned earlier the scrum master of the multi-project is group SW613F16
	\item[Product Owner and Security] \hfill \\
	The product owner of the multi-project is group SW614F16, they also have the areas of security, which means they will investigate any security measurements that are required for an application to be used by Aalborg municipality.
	\item[Unit and Integration Tests] \hfill \\
	This area of responsibility involves making a guide for the multi-project such that the other groups can easily write new unit tests for their tasks.
	They also have to verify that the tests made by groups are thorough enough.
	The group responsible for this is group SW615F16.
	\item[Graphics and Sound] \hfill \\
	The GIRAF Application Suite needs graphic icons, and may also require sound for the games in the application. 
	The group responsible for this is group SW617F16
	\item[Documentation and Wiki] \hfill \\
	This area requires establishing a standard of how documentation is done in the code, and responsible for the Wiki of the GIRAF project. We group SW618F16 are responsible for this areas.
	\item[Usability Tests] \hfill \\
	Performing usability tests on potential customers or with the customers themselves is important in order to know if the changes are working as intended, and to spot potential problems with the application. The group responsible for this is group SW619F16.	
\end{description}

\subsubsection*{Development and Organisational Tools}
Previously the project have used the project management tool Redmine.
However an opinion that is mostly consistent throughout all the papers from the previous semester is that Redmine should be replaced as it, according to last years students, is not a good tool.
Redmine offers tools for communication, calendars, documentation, issue tracking, a Wiki and more.
The primary issue with Redmine according to last years students refer to its lackluster communication solution.
As this is a problem stated throughout a majority of the papers the web application Slack is used as the communication tool between the groups involved in development.
Furthermore it was decided at the initial meeting to replace Redmine, with another software development tool: Phabricator, an open source, software development and project management platform. 
As Phabricator is still an unfamiliar tool issues may arise, however from the initial overview it seems to be geared towards agile development with applications specific to user stories and backlog tracking.
The workflow established for Phabricator is also very geared towards code review and documentation, something the groups agree seems very lackluster and should have more focus this semester.\footnote{For more information about Phabricator visit their website at http://phabricator.org/.}
Lastly with scrum meetings happening on a weekly basis a shared Google Docs folder contains all agendas and summaries from meetings.
Furthermore, while there is a primary rapporteur for each meeting, Google Docs allows everyone to pitch in for both the agenda and the summary which in turn means that if someone is not satisfied with the information in the summary, they can also add more information in real-time.
