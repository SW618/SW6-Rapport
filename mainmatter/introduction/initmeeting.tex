\section{Initial Multi-Project Meeting}
Once assigned\kim{Hvad menes med Once assinged? Hvem bliver assigned til hvad?} to the GIRAF project the papers written the previous year are distributed amongst the groups in order to accumulate some knowledge of what they did, what they may not have had time to do, how they worked and most importantly the evaluation thereof.
These papers provide a basis from which organisational tasks can be identified and assigned to groups involved in the development of GIRAF for this semester.

\section{Development Method and Organisational Tools}
The first meeting serves to address introduction\kim{Hvad betyder address introduction, jeg forstår det ikke.} of the most immediate organisational aspects such as version control, communication, development method, meetings and significant areas of responsibility.
For this meeting each group provides a list of the most significant pros and cons identified in each paper they are assigned.
Three papers per group in order to assure each paper are read at least twice\kim{Denne sætning er ikke færdig.}.
These pros and cons are used in consideration of development method and organisational tools.
Many of last years groups had also written directly to this years developers on the GIRAF project, these are also taken into consideration for this years GIRAF project. 

\subsubsection*{Development Method}
All the previous semesters of the GIRAF project used the agile development method scrum\kim{Det er ikke sandt. Dog er der altid blevet brugt Scrum of Scrum.}.
Scrum is an agile development method, which uses small teams with small sprints\kim{Hvis dette er første gang i nævner scrum så skal i lige give en kilde til den version af scrum som i følger.}. 
A product back-log lists the tasks which need to be done for the project. \kim{Overvej at skrive vigtige ord med kussiv første gang de nævnes, e.g. back-log. På den måde kan man nemmere finde definitationen af det pågældende term}
When a sprint starts\kim{comma} tasks from the product back-log are transferred to a sprint back-log and these tasks should then be completed before the end of the sprint.

Scrum uses a daily meeting called a daily scrum, its purpose is to inform the rest of the team of how each task is progressing, and if any problems are occurring.
A burndown chart is used to keep track of the progress for each sprint. 
Every task is time estimated, typically in man-hours, providing a total estimated work time when summed together. 
Every time a task is completed the burndown chart is updated so progress can easily be estimated.

A scrum board is also used which keeps track the sprint back-log, which tasks have not been started yet, which tasks are being worked on, which tasks are ready to be reviewed, and which tasks are done.
A couple of roles are normal in scrum, the product owner and the scrum master.
The product owner is the group member who controls the product back-log, it is their job to understand what the customer wants, and therefore how to prioritize the different tasks on the back-log.
The scrum master is the person who works to ensure that the development team uses scrum correctly, and helps the product owner organising\kim{spelling error} the product back-log.
For more information on scrum see the Scrum Alliance guide. \cite{scrum} \kim{Kilde skal stå før punktum.} \kim{Find en rigtig kilde på scrum, altså en bog. Dette gælder for alle jeres scrum referancer.}
\todo[inline]{Skal vi have mere om scrum eller er dette tilstrækkeligt? - Soren, jeg synes det et fint, Troels? - Tom. Vi skal måske gå lidt i dybden med hvilke af disse vi bruger. - Troels}
\kim{Det er lige meget med definitionen af scrum, i skal bare give en kilde til den definition som i bruger. Som Troels siger så det forklare hvilke scrum værktøjer som i bruger/ planlægger at bruge. Når i så ændre praksis i løbet af jeres sprints så kan i forklare hvad i har ændret. Lad os snakke mere om dette til vores møde.}

\bigskip
The multi-project from last year used a method called scrum of scrums, which attempts\kim{Dette (attempts) lyder som kritik af scrum of scrum, er det meningen?} to apply some principles of scrum in a larger group.
This works by having multiple groups, all using scrum, and then having ambassadors\ from every group meet and do the scrum meetings. 
Every group will perform a daily scrum individually, and the multi-group will perform a scrum meeting answering the same questions as one would at daily scrum but instead it is just the ambassadors from each group at this scrum meeting, whom then speak on the group's behalf.
The questions are as follows:

\begin{itemize}
    \item What did you do yesterday? 
    \item What will you do today?
    \item What obstacles are in your way or slowing you down?
\end{itemize}

These questions suggest that the meeting is to be held daily, though it is recommended to have the scrum of scrums two to three times a week instead. \kim{Er det virkelig de spørgsmål som i bruger på jeres scrum of scrum møder?}
Last year the students had a meeting two times a week, and some groups complained that this was too much, and this is agreed upon in the multi-project, so a single meeting once a week is suggested instead.
The recommendation of two to three times a week is based upon how scrum of scrum is used in the private sector according to Mike Cohn, president for Mountain Goat Software \citep{SCRUMoSCRUM}, however, as students we are not working on the GIRAF project all the time, we have courses which also require work, which leaves us with less time to work on the GIRAF project. 
With these arguments in mind it is decided that having the scrum of scrum once a week is reasonable.

The scrum of scrum meeting organisation can be seen on \myref{fig:scrumofscrum}.

\begin{figure}
\centering
\includegraphics[scale=0.4]{figures/scrumofscrum.png}
\caption{The structure of scrum of scrum. Figure from \cite{scrumofscrumfigure}.}
\label{fig:scrumofscrum}
\end{figure}

For the scrum of scrum a single product back-log with tasks is also created, each group in the multi-project then decides on tasks or user stories\kim{mangler definition (jeg har ikke været i stand til at finde nogen difinition i blant jeres kilder.)} from this back-log to put in their own internally kept sprint back-log.

The roles of the product owner and scrum master are considered to be areas of responsibility for certain groups in the multi-project rather than than the role of an individual, which will be explained next.

\subsubsection*{Areas of Responsibility}
The areas of responsibility are organisational and technical aspects of development that may require attention but are not necessarily directly linked to the software itself.
The use of scrum dictates that a project owner and a scrum master exists, as the scrum is composed of groups, these roles will be assigned to groups as areas of responsibility.
Furthermore the servers, the database and the automated building and testing tool Jenkins, has according to the papers read, often caused trouble so these will also serve as areas of responsibility. 
According to previous papers these groups also have a great deal of interaction, in order to reduce the time spent explaining the issue to other groups these areas will all be handled by the same group, as it happens all the areas is far too big a workload for a single group, thus two groups share these responsibilities.
From the previous papers it is evident that their work flow did not include a formal or even uniform way of documentation or even a consistent coding style.
As such documentation and unit/integration testing are also established as areas of responsibility.
We got the responsibility for one of these; documentation. 
This entails that we are to set up guidelines for the rest of the multi-project to follow, and to ensure they follow these guidelines.
Further areas of responsibility are derived largely from suggestions or positive feedback from the read papers, these include graphics, usability tests, security and social events/HR.
\info[inline]{jeg oververjer at stille alle op som punktform og forklare dem men ved ikke om det bliver for meget? - Marc. Jeg syntes en ``description'' kunne gøre godt her, så undgår man at det bliver rodet. - Troels Hvis vi gør det, skal det hele skrives om, og så forsvinder "historien" i teksten. - Søren}

\subsubsection*{Development and Organisational Tools}
Previously the project have used the project management tool Redmine.
However an opinion that is mostly consistent throghout all the papers from the previous semester is that Redmine should be replaced as it, according to last years students, is not a good tool.
Redmine offers tools for communication, calenders, documentation, issue tracking, a wiki and more.
The primary issue with Redmine according to last years students refer to its lackluster communication solution.
As this is a problem stated throughout a majority of the papers the web application Slack is used as the communication tool between the groups involved in development.
Furthermore it was decided at the initial meeting to replace Redmine, with another software development tool: Phabricator, an open source, software development and project management platform. 
As Phabricator is still an unfamiliar tool issues may arise, however from the initial overview it seems to be geared towards agile development with applications specific to user stories and backlog tracking.
The workflow established for Phabricator is also very geared towards code review and documentation, something the groups agree seems very lackluster and should have more focus this semester.
For more information about Phabricator visit their website at \cite{phabricatorWebsite}.
Lastly with scrum meetings happening on a weekly basis a shared Google Docs folder contains all agendas and summaries from meetings.
Furthermore, while there is a primary rapporteur for each meeting, Google Docs allows everyone to pitch in for both the agenda and the summary which in turn means that if someone is not satisfied with the information in the summary, they can also add more information in real-time.

\kim{Når man har cites der bare er links så fortrækker jeg at i bruger footnotes.}