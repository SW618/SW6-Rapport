\section{Contents of the GIRAF Application Suite}
This section will present the GIRAF application suite, which applications it contains and what they do.
Currently 11 applications are published on the Google Play Store under the developer name ``Giraf Autism Apps'', a link is available in \cite{GIRAFGOOGLEPLAY}.
There is a minimum set of applications that must be downloaded in order for anything to work, and others which are optional and only requires this set. 
This is due to the applications having originally been developed by different groups as individual small applications, rather than a collaborative work, this includes the launcher and administrative tools.
However it is recommended that all the applications is downloaded before use, in order to see the full picture of what the application suite offers.
\begin{description}
	\item[Giraf (Main Launcher)] The Giraf Launcher is meant to replace the default launcher of the device and is used to launch the applications of the Giraf Application Suite, but could also be used to launch other third party applications.
	Additionally the options menu and administration panel is opened from here. 
	\item[Administration Tool] The Administration Tool is primarily used by guardians to configure which applications each citizen can utilise. 
	\item[Sequence] Sequence (also known as ``Sekvens'' in danish) is a tool used to create sequences of pictograms for use in other parts of the application.
	These pictograms, which is a pictorial symbol of a phrase or word, is what the citizens sometimes communicate by, and this is also what helps them create these rigid plans, e.g. for going to the bathroom. 
	\item[Week Schedule] Week Schedule (also know as ``Ugeplan'' in danish) uses pictograms to display a schedule for each day of the week one week at a time. 
	\item[Pictosearch] Pictosearch is used to search through the pictograms such that they can be used for other applications. 
	\item[Pictoreader] The Pictoreader can display and read out-loud a series of pictograms. 
	\item[Category Manager] The Category Manager (also known as ``Kategoriværktøjet'' in danish) is used to manage categories of pictograms for use in other applications most notably the Category Game.
	It is also used to more easily be able to find certain pictograms, without having to use Pictosearch.
	\item[Category Game] The Category Game (also known as ``Kategorispillet'' in danish) is a game that asks the player to group up different pictograms into categories either predefined or defined in the Category Manager. 
	The game itself consists of a train which must drop off the pictograms at different stations according to their grouping.
	\item[Timer] The timer (also known as ``Tidstager'' in danish) is an application which can limit the amount of time another application is allowed to be active for. 
	This is often useful in combination with the Week Schedule, which can feature elements like, e.g., 10 minutes of playing the Category Game, or how long a third party application, e.g. a video game, could be open for.
	\item[Life Story] Life Story (also known as ``Livshistorier'' in danish) is similar to the Sequence application but is used to remember what actions were taken by a citizen throughout the day, or as a way of communicating since some citizens have a lot of trouble with their verbal communication.
	\item[Voice Game] Voice Game (also known as ``Stemme Spillet'' in danish) is a game in which the purpose is to control a car, either avoiding obstacles or gathering stars, using only your voice.
	This is supposed to help the citizens learn to control the volume of their voice, as some of troubles with either speaking too loud, and a few too silently
\end{description}}
\unsure[inline]{Screenshots?}
