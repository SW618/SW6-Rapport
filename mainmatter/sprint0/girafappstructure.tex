\section{Contents of the GIRAF Application Suite}

\info[inline]{Intro}
The Giraf Application Suite refers to the applications which have been developed for the use of citizens and guardians throughout the development of the GIRAF project. 
Currently 11 apps are published on the Google Play Store under the developer name ``Giraf Autism Apps'', a link is available in \cite{GIRAFGOOGLEPLAY}.
There is a minimum set of applications that must be downloaded in order for anything to work, and others which are optional and only requires this set. 
This is due to the apps having originally been developed by different groups as individual small apps, rather than a collaborative work, this includes the launcher and administrative tools
\unsure[inline]{More about this here ??? - I wrote a little more reasoning, is it enough? - Tom}
However it is recommended that all the applications is download before use. \todo{Kilde nødvendig? - ikke en kilde så meget som en begrundelse måske?(which i dont have) - Tom}
\begin{description}
	\item[Giraf (Main Launcher)] The Giraf Launcher replaces the default launcher of the device and is used to launch the applications of the Giraf Application Suite as-well as 3rd party applications. \todo{hvis AppSuite er alle GIRAF apps, hvilke 3rd party ting launches så med GIRAF launcheren?-Tom}
	Additionally the options menu and administration panel is opened from here. 
	\item[Administration Tool] The Administration Tool is primarily used by guardians to configure the applications each citizen can utilise. 
	\item[Sequence] Sequence (also known as ``Sekvens'' in danish) is a tool used to create sequences of pictograms for use in other parts of the application. 
	\item[Week Schedule] Week Schedule (also know as ``Ugeplan'' in danish) uses pictograms to display a schedule for each day of the week for a week at a time. 
	\item[Pictosearch] Pictosearch is used to search though the pictograms such that they can be used for other applications. 
	\info[inline]{Pictogrammer er brugt flere gange men egentlig ikke beskrevet hvad det er, burde vi eventuelt beskrive det? - Tom}
	\item[Pictoreader] The Pictoreader can display and read out-loud a series of pictograms. 
	\item[Category Manager] The Category Manager (also known as ``Kategoriværktøjet'' in danish) is used to manage categories of pictograms for use in other applications most notably the Category Game.
	\item[Category Game] The Category Game (also known as ``Kategorispillet'' in danish) is a game that asks the player to group up different pictograms into categories either predefined or defined in the Category Manager. 
	The game itself consists of a train which must drop off the pictograms at different stations according to their grouping.
	\item[Timer] The timer (also known as ``Tidstager'' in danish) is an application which can limit the amount of time another application is allowed to be active for. 
	This is often useful in combination with the Week Schedule, which can feature elements like, e.g., 10 minutes of playing the Category Game. 
	\item[Life Story] Life Story (also known as ``Livshistorier'' in danish) is similar to the Sequence application but is used to remember what actions were taken by a citizen throughout the day. 
	\item[Voice Game] Voice Game (also known as ``Stemme Spillet'' in danish) is a game in which the purpose is to control a car, either avoiding obstacles or gathering stars, using only your voice. 
\end{description}}
\unsure[inline]{Screenshots?}
