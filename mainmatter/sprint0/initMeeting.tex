\section{Initial Multi-Project Meeting}
Once assigned to the GIRAF project the papers written the previous year are distributed amongst the groups in order to accumulate some knowledge of what they did, what they may not have had time to do, how they worked and most importantly the evaluation thereof.
These papers provide a basis from which organisational tasks can be identified and assigned to groups involved in the development of GIRAF for this semester.

\subsection{Development Method and Organisational Tools}
The first meeting serves to address introduction of the most immediate organisational aspects such as version control, communication, development method, meetings and significant areas of responsibility.
For this meeting each group provides a list of the most significant pros and cons identified in each paper they read.
Three papers per group in order to assure each paper are read at least twice.
These pros and cons are used in consideration of development method and organisational tools.
Many of the groups had also written directly to the next year of developers on the GIRAF project, these are also taken into consideration for this years GIRAF project. 

\subsubsection*{Development Method}
All previous semesters of the GIRAF project has used the agile development method Scrum.

Scrum; Scrum of Scrums; Previous failures(distinct split); Our group also uses Scrum; Scrum of Scrum of Scrums

\subsubsection*{Areas of Responsibility}
The areas of responsibility are organisational and technical aspects of development that may require attention but are not necessarily directly linked to the software itself.
The use of Scrum dictates that a project owner and a Scrum master exists, as the Scrum is composed of groups, these roles will be assigned to groups as areas of responsibility.
Furthermore the servers, the database and the automated building and testing tool Jenkins, has according to the papers read, often caused trouble so these will also serve as areas of responsibility. 
According to previous papers these groups also have a great deal of interaction, in order to reduce the time spent explaining the issue to other groups these areas will all be handled by the same group, as it happens all the areas is far too big a workload for a single group, thus two groups share these responsibilities.
From the previous papers it is evident that their work flow did not include a formal or even uniform way of documentation or even a consistent coding style.
As such documentation and unit/integration testing are also established as areas of responsibility.
We got the responsibility for one of them; documentation. \info[inline]{Måske vi skal skrive mere om hvad dette indebærer for os, eller skal det blot komme senere? - Troels}
Further areas of responsibility are derived largely from suggestions or positive feedback from the read papers, these include graphics, usability tests, security and social events/HR
\info[inline]{jeg oververjer at stille alle op som punktform og forklare dem men ved ikke om det bliver for meget? - Marc. Jeg syntes en ``description'' kunne gøre godt her, så undgår man at det bliver rodet. - Troels}

\subsubsection*{Development and Organisational Tools}
Previously the project have used the project management tool Redmine.
However an opinion that is mostly consistent throghout all the papers from the previous semester is that Redmine should be replaced as it, according to last years students, is not a good tool.
Redmine offers tools for communication, calenders, documentation, issue tracking, a wiki and more.
The primary issue with Redmine according to last years students refer to its lackluster communication solution.
As this is a problem stated throughout a majority of the papers the web application Slack is used as the communication tool between the groups involved in development.
Furthermore it was decided at the initial meeting to replace Redmine, with another software development tool: Phabricator, an open source, software development and project management platform. 
As Phabricator is still an unfamiliar tool issues may arise, however from the initial overview it seems to be geared towards agile development with applications specific to user stories and backlog tracking.
The workflow established for Phabricator is also very geared towards code review and documentation, something the groups agree is very lackluster and should have more focus.
For more information about Phabricator visit their website at \cite{phabricatorWebsite}.
Lastly with meetings happening on a weekly basis\info{which ones? Har vi introduceret weekly scrum i Sørens eller hvad? - Troels} a shared Google Docs folder contains all agendas and summaries from meetings.
Furthermore, while there is a primary rapporteur for each meeting, Google Docs allows everyone to pitch in for both the agenda and the summary which in turn means that if someone is not satisfied with the information in the summary, they can also add more information in real-time.
