\section{Initial Meeting (This headline clearly sux donkey balls)}
Once assigned to the GIRAF project the papers written the previous year are distributed amongst the groups in order to accumulate some knowledge of what they did, what they may not have had time to do,  how they worked and most importantly the evaluation thereof.
These papers provide a basis from which organisational tasks can be identified and assigned to groups involved in the development of GIRAF for this semester.

\subsection{Development Method and Organisational Tools}
The first meeting serves to address introduction of the most immediate organisational aspects such as version control, communication, development method, meetings and significant areas of responsibility.
For this meeting each group provides a list of the most significant pros and cons identified in each paper read, three papers per group in order to assure each paper was read at least twice.
These pros and cons are used in consideration of development method and organisational tools.

\subsubsection*{Development Method}
All previous semesters of the GIRAF project has used the agile development method Scrum.

Scrum; Scrum of Scrums; Previous failures(distinct split); Our group also uses Scrum; Scrum of Scrum of Scrums

\subsubsection*{Areas of Responsibility}
Areas of responsibility are organisational and technical aspects of development that may require attention but are not necessarily directly linked to the software itself.
The use of Scrum dictates that a project owner and a Scrum master exists, as the Scrum is composed of groups, these roles will be assigned to groups as areas of responsibility.
Furthermore the servers, the database and Jenkins, the automated build, has, according to the papers read, often caused trouble so these will also serve as areas of responsibility.
According to previous papers these groups also have a great deal of interaction, in order to reduce the time spent explaining the issue to other groups these areas will all be handled by the same group, as it happens all the areas is far too big a workload for a single group, thus two groups share these responsibilities.
From the previous papers it is evident that their work flow did not include a formal or even uniform way of documentation or even code.
As such documentation and unit/integration testing are also established as areas of responsibility, documentation being the one we are responsible for.
Further areas of responsibility are derived largely from suggestions or positive feedback from the read papers, these include graphics, usability tests, security and social events/HR
\info[inline]{jeg oververjer at stille alle op som punktform og forklare dem men ved ikke om det bliver for meget?}

\subsubsection*{Development and Organisational Tools}
Previously the project have used the organisational tool Redmine which is a project management tool.
However an opinion that is mostly consistent throghout all the papers from the previous semester is that Redmine should be replaced as it, according to last years students, is not a good tool.
Redmine offers tools for communication, calenders, documentation, issue tracking, a wiki and more.
The primary issue with Redmine according to last years students refer to its lackluster communication solution.
As this is a problem stated throughout a majority of the papers the web application Slack is used as the communication tool between the groups involved in development.
Furthermore it was decided at the initial meeting to replace Redmine, with another software development tool, for which the suggestion was Phrabricator, an open source, software development platform.
As Phrabricator is still an unfamiliar tool issues may arise however from the initial overview it seems to be geared towards agile development with applications specific to user stories and backlog tracking.
The workflow established for Phrabricator is also very geared towards code review and documentation, something the groups agree is very lackluster and should have more focus.
Lastly with meetings happening on a weekly basis a shared Google Docs folder contains all agendas and summaries from meetings.
Furthermore, while there is a primary rapporteur for each meeting, docs allows everyone to pitch in for both the agenda and the summary which in turn means that if someone is not satisfied with the information in the summary, they themself can add more information.
