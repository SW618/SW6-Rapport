\section{Initial Meeting (This headline clearly sux donkey balls)}
Once assigned to the GIRAF project the papers written the previous year are distributed amongst the groups in order to accumulate some knowledge of what they did, what they may not have had time to do,  how they worked and most importantly the evaluation thereof.
These papers provide a basis from which organisational tasks can be identified and assigned to groups involved in the development of GIRAF for this semester.

\subsection{Development Method and Organisational Tools}
The first meeting serves to address introduction of the most immediate organisational aspects such as version control, communication, development method, meetings and significant areas of responsibility.
For this meeting each group provides a list of the most significant pros and cons identified in each paper read, three papers per group in order to assure each paper was read at least twice.
These pros and cons are used in consideration of development method and organisational tools.

\subsubsection*{Development Method}
All previous semesters of the GIRAF project has used the agile development method Scrum.

Scrum; Scrum of Scrums; Previous failures(distinct split); Our group also uses Scrum; Scrum of Scrum of Scrums

\subsubsection*{Areas of Responsibility}
The use of Scrum also dictates that a Project Owner and a Scrum Master exists, as the Scrum is composed of groups, these roles will be assigned to groups as areas of responsibility.
Furthermore the servers, the database and Jenkins, the automated build, has, according to the papers read, often caused trouble so these will also serve as areas of responsibility.
According to previous papers these groups also have a great deal of interaction, in order to reduce the time spent explaining the issue to other groups these areas will all be handled by the same group, as it happens all the areas is far too big a workload for a single group, thus two groups share these responsibilities.

\subsubsection*{Development and Organisational Tools}

