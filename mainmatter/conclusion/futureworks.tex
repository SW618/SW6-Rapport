\chapter{Future Work on GIRAF}\label{chp:futureworks}
In this chapter we present the immediate work that we believe would be most beneficial to continue development on GIRAF.
This chapter will not be a complete list of all the work to be done, rather it will be contained to the areas of GIRAF where we have been involved in throughout the projects, this means we will discuss improvements in regards to Picto Search, the Week Schedule App, the GIRAF Launcher and of course the REST API.
The REST API affects the future of most apps including the ones we already mentioned, as such we will start by explaining what is missing from the REST API as well as argue for why we believe that the REST API should be prioritised.

This list is not solely compiled from our point of view, it is the result of a final meeting with representatives from each group at which we spoke of the degree to which we had achieved our goals as well as what we believed would be best for the development of GIRAF to continue with.
%Limit this to what we have been involved in, goes beyond this chap

\section{REST API}
With the REST API synchronisation will become a part of GIRAF.
This is something that we believe is mandatory before GIRAF could ever be announced as even close to finished.
Despite being discouraged by the previous developers telling us to ignore synchronisation as it would not be reachable within one semester, we believe it to detrimental to the quality and usability of GIRAF and as such the REST API should be the primary goal for the following semester.
It is worth noting here that simply because we find this to be the most important other things should not be neglected, a lot of the REST API is internally blocking and requires plenty of collaborative work between groups, as such putting too many developers may slow down the development.

The areas that yet need to be worked on for the REST API are login, endpoints, client-side implementations and web administration.
Our group has not been involved in developing the new login system and thus we will not discuss that further, web administration is yet untouched so for that we will simply explain why it is needed and how it relates to REST. 
\subsection{Endpoints}
While a number of endpoints and their models have been finished this year, yet more remain before the current applications can rely solely on the REST API for data.
Currently endpoints exist for users, departments, the derivatives of frames and week schedules.
This however is not enough as still more data exists for other applications in GIRAF, namely maps and game settings for Voicegame and Categorygame, as well as categories themselves.
Further than that administrative data such as user settings is also not yet available through the REST API.
All of these must be constructed before GIRAF can fully migrate to use the REST API rather than the current database.
\subsection{Client-Side Implementation}
Once all the endpoints are completed, or before that even, client-side implementations must be made in order to actually utilise the REST API.
In general one primary concern remains here which for us have not yet been discussed as it has not been important.
Either some module that translates data must be developed, or the apps must be refactored as to use the same models as the REST API.
An example here would be the Week Schedule model and endpoint that we have developed.
Our model uses a class Week Schedule with and enummap to map days, where the app currently uses nested sequences.
Either the data retrieved and sent to the REST API must enter a translation process or the app should undergo a refactoring process as to use something that more closely resembles the model used by the REST API.
The reason for the difference is both that we were dissatisfied with the degraded model used by the app and thus decided to improve the model to retain a high code quality in the REST API.
As for choosing between one or the other, refactoring may take longer and be the best long term solution, while a translation module may be a faster solution, but worse in the long term and make it harder for future developers to use.

Aside from this, client-side must also consider version control, particular in regards to offline mode and here implement either the solution that we decided would be best in \myref{ssec:policy}, or decide upon another way of handling this issue, either way it must be handled with the use of the REST-API
%Perhaps make dummies for the not yet developed endpoints in regards to apps so app developers can tell  endpoint developers what data they need. not an issue for us as we had been developing on the apps we did endpoints for.

\subsection{Web Administration}

\section{Week Schedule}

\section{Launcher}

\section{Picto Search}


%PS
    %Search Improvemens
        %Better algorithm
        %Sorting is currently after ID, retard implemented this
        %Paging/Lazyloading
        %Kategory and Sequences as results
        %REST API implement, this would allow easy sequence implementation.
        %Advanced search?(Comes with Sequence/Kategory implementation IMO)
%WS
    %GoalsWeSet
        %Unfinished features
        %TryOutRdy
    %Future
        %Changes to progress, REST API have already considered this
        %Responsive, the app feels a little slow, Rest load types can affect this
        %Rest implemented Client-Side
        %Vælg Antal is placed oddly and is very unspecific, took months for us to figure it out
        %The app is a mess of Sequences, a refacter with REST may be extemely beneficial for future development.
%REST
    %Apps
        %Administration
        %PictoCreater
        %VoiceGame
        %CategoryGame
            %Categories
        %Compartmentalize this ya mofo!
    %Logon stuff???
    %Deploy and Migrate DB
    %Use what we developed, Client-Side adapting
    %Translater Interface between apps and APIS OR restructure apps
    %WebAdmin will use API as well Prioritised just after REST itself

%Launcher
    %Offline
    %Kinda useless atm, any changes will have to change with REST Anyway
    %Stability, the asynch internet checking process can cause crash if WiFi is shut down without being logged in
    %Rest Logon to be finished
    %Offline rest considerations - use lastEdit and stuff perhaps make a helper class to help with this for all apps? GENERALIZE IT LIKE A MOTHERFUCKER!
    %Security and cryptography
    %Synchronisation
    %Core packaging
