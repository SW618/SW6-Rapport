\chapter{Project Evaluation}
\section{Project Summary}
%Multi Goals
At the very beginning of the project our problem was very broad and undefined, improve GIRAF by stopping the numerous crashes caused by half system, let the system work offline and let some apps be launched without going through the launcher app.
This let to the first sprint mainly being bug and UI fixes that had been revealed at the previous years' user tests, held at the very end of their semester.
At the beginning of the second sprint we decided to focus our efforts.
Realising that spreading out to fix various bugs in all the apps would not get us far, we held a customer meeting and with the information from the customers decided what apps we should focus on.
These apps were Week Schedule, VoiceGame, PictoReader and Launcher which meant offline mode.
While these apps were the focus, we decided to start on a REST API as well.
From our very first meeting with last years developers we had been told that they had tried to implement synchronisation and that it worked, they also mentioned that previous years had tried to do the same, yet never finished.
As it happens their solution did not work, leaving GIRAF with no synchronisation.
This led to us also developing on the REST API aside from our primary apps.
%Our Goals
\bigskip \noindent
At the beginning of the third sprint our group started to focus on the REST API by producing endpoints.
The goal was not to finish the REST API, but to finish enough endpoints, with a high enough code quality and with guides for next year, such that they could continue developing on the REST API, as it was far too big a task for us to do in a single semester.

With this we were successful, the REST API contains very structured code and numerous guides have been created for next years developers which should help them quickly set up a build environment and understand how far REST development is, and what should be done next.
As for the overall project goals we managed to finish VoiceGame and make it standalone, Week Schedule has seen significant progress satisfying the customers need, however without synchronisation its usefulness is low, the launcher allows offline mode, although after an initial launch technically one is always in offline mode as no web communication is actually used.
PictoReader has not been as successful as the development of other apps and still requires some fixing.

\section{Multi--Project Development}
%Multi--Project
This semester we have been working on GIRAF as part of a Multi--Project group.
Combined 32 people divided into 9 groups have been working on GIRAF, with this many developers this of course led us to use some project management methods.
The project has been run using parts of Scrum, for all groups this means a Scrum of Scrums has been the development method as all group have used Scrum in their groups as well, or at least that is what have been said at the meetings.
Through the development of the project, our development method developed.
After every sprint realisations about what went badly in a sprint was addressed.
For this we used a method we called ``Stop, Start, Continue'', as the name suggests every group would mention what they wanted us to stop doing, start doing and keep doing.
Through these incremental improvements we went from Sprint 1, a somewhat unorganised sprint of people being unfamiliar with the work flow, and simply using user stories from the previous semester, badly defined ones at that, to a point where most user stories were defined correctly, i.e. As someone i want something such that some purpose.

By each sprint code quality also improved and less breaking changes were made, this is in large part due to our code review structure which initially was either ignored or simply not understood fully, however in the end, while it was not flawless, significantly improved the code quality before it was pushed.
However, the most important thing that we learned during the Multi--Project was not to be too rigid in our development.
An example of this would be user stories, technically they are the responsibility of the Product Owners, yet conforming to this creates a bottleneck on user stories, as such we decided to both rephrase old user stories to the proper format, and create new ones.
There was no downside in doing this, it simply removed a bottleneck through relaxing the corporate thinking and as such it was a good decision.
All in all the multi--project development may not have started out perfect, but by the third and fourth sprint our development method was solid, and as can be seen by the work we managed to complete, yielded good results.

\section{Internal Development}\todo{im no fan, some1 improve -M}
%Our development method
As mentioned we have in our own group also utilised parts of Scrum.
We have made use of daily scrum meetings and a physical scrum board to represent our sprint backlog,
The physical Scrum board gives us some idea of how we are doing according to our time schedule, whether we are ahead and need more tasks as was the case in sprint 1, or whether we are behind and need to perhaps drop a task, or simply put more work into the it, the latter being the favored in our group.
The daily Scrum meetings have let us have an insight into what each individual has been working on, such that any group member always knows what is currently being worked on, what has finished, who may need help with something as well as what tasks or user stories have yet to be started.
Furthermore using Scrum, as well as all other groups using Scrum, has allowed for our Scrum of Scrums multi--project structure, this way both the groups individually and the multi--project as a whole fit together, the shared backlog and sprint lengths being remarkably useful.

During development we attempted a variety of methods to help our development, as well as fixing things we had been neglecting.
While we have made use of the Scrum board, it has not always been to the full effect, it started off well, however after a while, during the second sprint, it became less used leading to a slightly more messy sprint with less communication in the group.
This was changed the following sprints where we worked on the REST API where communication in the group was fine, however another method from XP was adapted here, pair programming.
Pair programming was both successful and unsuccessful.
The group split into two pairs, the one pair successfully used pair programming, where as the other pair was unsuccessful and in the end stopped the pair programming as it was losing too much time, for no gain.
This showed us that we should adjust to what fits our group, as such the successful pair continued with pair programming, while the other pair left it behind.
For the unsuccessful pair it did not mean not working together, merely that pair programming was not an option, and as such the pair split and rather than using pair programming, merely posed inquiries to one another when needed.

\section{GIRAF Development}\todo{vi snakkede om at quote kunderne men kan ikke huske på hvad -M}
%Eval of our work
Most of our work has been focused on the REST API, something that has no user evaluation or feedback whats so ever, as it is not interesting nor important for them to know about, therefore we will evaluate the REST API not in regards to potential application, but rather whether or not we deem that the REST API is sufficiently documented such that the following semester can continue working on it.
Also the RESP API is not the only thing we have been working on, as such we will evaluate the larger changes that we have made that affects the users and have been shown, namely changes to PictoSearch and Week Schedule.

%PS and WS
For PictoSearch we improved the search tool greatly.
We provided PictoSearch with a more responsive search feature, the customers no longer feel uncertainty whether or not the search is on going.
The more responsive search gives the user the illusion that the search is also faster, the search of course could also be made faster, which is suggested for the next years developers to look into.
As for the Week Schedule, quality of life features were our goal here, particularly scrolling and dragging in such a way that it felt familiar and natural.
This was achieved by making the app similar to standard Android and iOS apps, using long press to discern the two.
At our very last customer meeting, a review of the overall work, one of the customers mentioned in particular that Week Schedule had seen significant improvement and perhaps soon was ready to be tried.
Considering this comment, we would say that Week Schedule has been exceedingly successful, this of course not only being our own work, but a combined effort by the multi--project.
%REST
\bigskip \noindent
As for the REST API, the important thing for us here is that the developers that are to take over next year, can continue our work.
For this reason structure in the API has been very important for us.
However not only the code must be structured, it must be evident, what technologies and tools we use, what they do and how to use them.
In order to achieve this we created a number of guides.
While it is difficult for us to evaluate the success of these guides as we have been working on the API for a while, we do know that following the guides to setup build environment, works, and can be done in a manner of minutes which we consider to be quite successful.
Furthermore the code contains a very clear structure of core, persistence and service, all of which are explained, and have a dedicated step by step guide on how to develop each layer.
This combined with the endpoints we have developed, we believe to be sufficient to at the very least understand what we have done, and attempt to recreate it even if they should require outside help, there should be enough information to get them started.

Overall the group have been very committed to ensure not only that the REST API may be developed the following semester, but also that GIRAF is improved and stabilised such that at some point, it may see actual use by satisfied customers.
Furthermore the customers seem satisfied by the work done throughout this semester by the entire multi--project, and since the work this refers to is what we set out to do, we consider the semester a success.