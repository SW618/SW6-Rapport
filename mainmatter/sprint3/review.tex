\chapter{Sprint End}

\section{Sprint Review}\label{sprintreview3}
\todo{depends on Marcs rewrite}

\section{Sprint Retrospective}
This section will be a sprint retrospective separated into two parts, a part for the multi project and one internally in the group.

\subsection*{Multi Project Retrospective}
The multi project is running smoothly now and as such it is minimal what changes we are doing to the process.
It was discussed that people left quickly at the weekly scrums and because of this sometimes when another group had planned to talk with someone, the someone had left the room before they had a chance to do so.
Therefore, we decided that having the scrum master ask if we needed to speak with someone at the scrum meeting would be great such that they do not leave as soon as possible.

One last thing was that the differentials which are uploaded to Phabricator sometimes became too big ~1000 lines of code, which makes it very difficult to review in a reasonable amount of time.
To make it easier to grasp what a differential is about we decided that they should be smaller and removed the rule that a differential had to be a fully completed user story, as long as the changes does not break the application.

\subsection*{Internal Project Retrospective}
This sprint we made sure to have a scrum meeting daily, and we also used the scrum board at these meetings, which had earlier been neglected.
This gave everyone in the group a larger overview of our progress, and even helped us produce more work.
The user stories for this sprint have been more difficult than the other sprints but we are happy with our results.
We used pair programming for every single user story and only worked on the programming assignments when at the university while other tasks like reviewing code or writing for the report was done at home.
This greatly improved the quality of our work and also helped uur productivity because no one was doing nothing at any one time.
Pair programming for the REST API is also especially relevant as the code is more complex than the application code which we have previously written, the design is more thorough and the pair programming aspects brings much more discussion of the design.

We will continue to pair program through sprint 4 and also utilize the daily scrum and scrum board, as we all agree that this is the most succesful sprint yet, even though we did not finish the user stories but the reasoning for this has been explained in \myref{sprintreview3}.

The next part is the last sprint of this project where we will finish the service layers for both the Sequence and WeekSchedule end points.