\chapter{Sprint Planning}
This chapter will introduce the work that is to be done in the sprint along with a reasoning for why these user stories and tasks were chosen.
%Tasks for the sprint

\section{User stories \& Tasks}
\begin{description}[style=unboxed]
    \item[{[}\phigh{]} As a guardian I would like the launcher to tell me how to add applications if none are active, such that it is easier to add applications for beginners.] \hfill \\ 
    Currently there is no help in GIRAF to gain knowledge of how to add an application for one self or for other users into the GIRAF launcher.
    When you first open GIRAF you are faced with an empty screen where you hopefully figure out that you need to go into settings and add these applications, even to guardians, and also to the citizens.
    This can be changed in a number of ways, eg. adding applications to guardians on start up and/or creating a tutorial which shows the users how to add applications.
    The user story has been estimated to take 5 EP.
\end{description}
The other work we will be doing this sprint comes from the fact that there is no security in the database.
They are not created as user stories because they stem from user stories themselves, and also because they are only back-end.
One of the user stories are as follows:

\textbf{\textit{As a user I would like to have my personal information secure such that other users cannot see my information.}}

As of now all personal information is downloaded to every single tablet when you first start the GIRAF launcher.
There is also no working synchronization of data at the moment. 
So if a week schedule is added to a user on one tablet, the week schedule will not be present on other tablets, and as we have been told by the customers that the citizens do not always use the same tablet this causes a problem.
Therefore it has been decided by the multi project in sprint 1 that a REST-API will help with creating privacy using a different login system, aswell as creating a way to synchronise the data on the tablets.
Group SW615 has now asked for help as the task of creating a new API along with a new database for it is too big a task for one group.
Why a REST-API is created will also be further explained in \myref{sec:current}.
We have chosen to help with this task, and will therefore be completing the following API endpoints for this sprint: 

\begin{description}[style=unboxed]
	\item [Endpoint for Pictograms] The endpoint to retrieve pictograms, this will be used in the PictoSearch library.
	Creating this endpoint is estimated at 13 EP.
	\item [Endpoint for Sequences] The endpoint to retrieve Sequences, this will be used to retrieve Sequences in e.g. the Sequence application.
	Creating this endpoint is estimated at 13 EP.
	\item [Endpoint for Week Schedules] This is the endpoint to retrieve week schedules from the database.
	Creating this endpoint is estimated at 31 EP.
\end{description}
Further descriptions and analysis will be given in a section for each of these end-points.

We will be collaborating on these tasks along with group SW615 throughout the sprint, both in the design process but also in implementing the different parts of the API.
The following chapter will further investigate the problems with the current database system, and give motivation for creating a new system using a REST-API.

