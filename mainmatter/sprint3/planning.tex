\chapter{Sprint Planning}
This chapter will introduce the work that is to be done in the sprint along with a reasoning for why these user stories and tasks were chosen.
%Tasks for the sprint

\section{User stories \& Tasks}
\begin{description}[style=unboxed]
    \item[{[}\phigh{]} As a guardian I would like the launcher to tell me how to add applications if none are active, such that it is easier to add applications for beginners.] \hfill \\ 
    Currently there is no help in GIRAF to gain knowledge of how to add an application for one self or for other users into the GIRAF launcher.
    When you first open GIRAF you are faced with an empty screen where you hopefully figure out that you need to go into settings and add these applications, even to guardians, and also to the citizens.
    This can be changed in a number of ways, eg. adding applications to guardians on start up and/or creating a tutorial which shows the users how to add applications.\todo{ADD estimations}
\end{description}
The other work we will be doing this sprint comes from the fact that there is no security in the database.
All personal information is downloaded to every single tablet when you first start the GIRAF launcher.
There is also no working synchronization of data at the moment. 
So if a week schedule is added to a user on one tablet, the week schedule will not be present on other tablets, and as we have been told by the customers that the citizens do not always use the same tablet this causes a problem.
Therefore it has been decided back in sprint 1 by group SW615 that a REST API will be able to solve these issues, creating privacy with a different login system, along with synchronising the data on the tablets.
Group SW615 has now asked for help as the task of creating a new API along with a new database for it is too big a task for one group.
Why a new database is created will also be further explained in INDSÆT MYREF HER !!.
We have chosen to help with this task, and will therefore be completing the following API endpoints for this sprint: 

\begin{itemize}
	\item Endpoint for Pictograms
	\item Endpoint for Sequences
	\item Endpoint for Week Schedules.
\end{itemize}

We will be collaborating on these tasks along with group SW615 throughout the sprint, both in the design process but also in implementing the different parts of the API.
The following section will further investigate the problems with the current solution, and why we have chosen not to try and expand on this solution, and rather create a new from the bottom.

\section{The current database system}
%Different from other sprints because of the API
%Working together with group SW615 for the API as they started the task last sprint
%Will collaborate a lot
%
%What is the problem with the current solution?
%What are hte possible problems of designing the API.
%What can the API solve for the project
%
