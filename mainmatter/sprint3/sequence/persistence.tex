\subsection{Persistence --- Querying the Database}
In this section the process of querying the database in relation to sequences will be presented.
Two DAOs will be introduced and we will explain why the \texttt{Choice} objects does not have a Data Access Object.

\subsubsection{Frame DAO}
Every sequence is, as previously described, essentially just an ordered list of \texttt{Frame}s alongside some metadata, such as the title of the sequence and so forth.
Normally when handling \texttt{Sequence}s in the REST API, this list of \texttt{Frame}s is accessed as an ordinary \texttt{List} structure in Java.
However when a \texttt{Sequence} object is instantiated or the elements, i.e. the list of \texttt{Frame}s, is changed, the desired \texttt{Frame}s must first be found, before they can be added to the list and Hibernate can establish the correct relations.
Because of this retrieving \texttt{Frame} objects from the database is an essential procedure.

Moreover after a \texttt{Frame} object have been retrieved from the database, because \texttt{Choice}, \texttt{Sequence} and \texttt{Pictogram} all extend the \texttt{Frame} class, Hibernate is able to find the specific object using the foreign keys.
For Hibernate to do this the \texttt{Frame} class has the annotation: \texttt{@Inheritance(strategy = InheritanceType.JOINED)}.

\bigskip
The \texttt{FrameDaoImpl} which is the Hibernate implementation of the interface \texttt{FrameDao}, has one method for retrieving \texttt{Frame}s.
The method is named \texttt{byId} and behaves precisely as the method of the same name in the \texttt{PictogramDaoImpl} class as described in \todo{ref!!}.

\subsubsection{Sequence DAO}
The most significant difference between the aforementioned \texttt{FrameDaoImpl} and the \texttt{SequenceDaoImpl} is the ability to retrieve \texttt{Sequence}s by querying by title.
This is done with the method which can be seen in \myref{lst:seqdao_search}.
Here a \texttt{TypedQuery} is used with a \texttt{LIKE} clause, which matches the lowercase modification of the \texttt{Sequence} titles to a given string, as seen on line~\ref{lst:seqdao_search_like}.
The string which the title is matched against is likewise transformed to lowercase, such that searches are not case sensitive.
Moreover the string is padded with the \texttt{\%} wild card operator, which matches one or more characters, thus making the search effective on substrings.
These two transformations can be seen on line~\ref{lst:seqdao_search_string}.

\begin{lstlisting}[float, floatplacement=h, caption={The method in \texttt{SequenceDaoImpl} which allows for \texttt{Sequence}s to be searched for by title}, label={lst:seqdao_search}]
public Collection<Sequence> searchByUserAndTitle(User user, String title) {
    TypedQuery<Sequence> query = em.createQuery(
            "SELECT s " +
            "FROM Sequence s " +
            "WHERE (LOWER(s.title) LIKE :title) " +(*@\label{lst:seqdao_search_like}@*)
                "AND (s.accessLevel = :public " +
                    "OR (s.accessLevel = :protected " +
                        "AND  s.department = :usersdepartment) " +
                    "OR (s.accessLevel = :private " +
                        "AND s.owner = :user))",
            Sequence.class);
    query.setParameter("title", "%" + title.toLowerCase() + "%");(*@\label{lst:seqdao_search_string}@*)
    query.setParameter("public", AccessLevel.PUBLIC);
    query.setParameter("protected", AccessLevel.PROTECTED);
    query.setParameter("private", AccessLevel.PRIVATE);
    query.setParameter("user", user);
    query.setParameter("usersdepartment", user.getDepartment());
    return query.getResultList();
}
\end{lstlisting}

\subsubsection{Retrieving Choices}
In the case of \texttt{Choice}s and the way they are used in the REST API and GIRAF, a DAO for retrieving them from database is useless.
This is due to the fact that \texttt{Choice}s always reside on a \texttt{WeekSchedule} or on a \texttt{Sequence}, and should not be searched for as e.g. \texttt{Sequence}s or \texttt{Pictograms}.
Therefore we have not implemented a \texttt{ChoiceDAO}, as Hibernate facilitates the retrieval of \texttt{Choice}s through other objects or the Frame DAO.
