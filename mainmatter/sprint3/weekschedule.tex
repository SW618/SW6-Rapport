\section{REST API: Week Schedule}\label{sec:restws}
This section pertains to the following user story: \\
\userstory{As a developer I would like an endpoint for Week Schedules, such that I can retrieve them from GIRAF.} \\
Specifically the two sub tasks regarding the core layer and the persistence layer.

For the sake of this section, we refer to both the classes and also the real life objects that they represent, as such we make a distinction between how the two are written.
When referring to it in the sense o
For when a class is referred to we write it as the class name and in a specific font, e.g. \texttt{WeekSchedule}.
If we refer to the object that it identifies in the real world it will be written in standard language grammar with our normal font, e.g. week schedule.

\noindent
One of the focus points for this years GIRAF project have been the Week Schedule app.
The data for this application should be shared across devices, and some information across users.
Therefore it is crucial that the Week Schedule part of GIRAF is also part of its REST API.
This Section will explain the parts of the Week Schedule REST API we developed during this sprint; the Core (model) and Persistence. 

\subsection{Model}
In this subsection we will introduce the model we made for the Week Schedule app.
First we will explain what this is meant to be used for and how it is currently modelled, that is the Week Schedule app, then we will explain our solution to the problem. 

In the Week Schedule app the view is split into seven views, each representing a day of the week.
In the current version this is modelled as the Week Schedule being a sequence containing seven sequences.
Each of these views contains an ordered list of a combination of pictograms, sequences and choices.
As explained earlier we model these as the \texttt{Frame}--superclass, from which they all inherit. % They inherit despite noone dying. 
Each frame in a week schedule additionally have a number of boolean values that determine their progress, which indicates the status of the frame. 
The booleans that currently exist denote the progress are: Complete, Active, NotComplete and Canceled. 
As mentioned this progress is not stored on the server, but on a locally saved file.
Since the progress information is personal, each user needs to separately have this information stored. 
As we believed some of the design choices to be odd, for our own design choices we decided to base it upon how the Week Schedule app is used, and disregard the current model.

\subsubsection{Week Schedule}
The table seen in \myref{tbl:WeekSchedule} provides an overview and a short description of how we model a \texttt{WeekSchedule} in the REST API.
For some of the fields further details and thoughts behind their creation is provided.
The \texttt{WeekSchedule} class is central to the modelling of week schedules.
All information about a week schedule can be derived from this class, we will start by presenting this class and then following its fields to find the \texttt{Weekday}, \texttt{WeekdayFrame} and lastly \texttt{WeekdayFrameProgress} classes.

\begin{table}[ht]
\resizebox{\textwidth}{!}{%
    \centering
    \begin{tabular}{lll}
        \multicolumn{3}{c}{\Large{Class: \texttt{WeekSchedule}}}                                                                                \\
        \tblgrpsep
        \tblgrpsep
        \textbf{Field Name} & \textbf{Type}                                     & \textbf{Short Description}                                    \\
        \midrule
        \texttt{id}         & \texttt{long}                                     & Unique Identifier                                             \\
        \texttt{name}       & \texttt{string}                                   & Name of the \texttt{WeekSchedule}                             \\
        \texttt{thumbnail}  & \texttt{Pictogram}                                & The pictogram shown as thumbnail                              \\
        \texttt{lastEdit}   & \texttt{Date}                                     & Timestamp for when the \texttt{WeekSchedule} was last edited  \\
        \texttt{department} & \texttt{Department}                               & The department that owns the \texttt{WeekSchedule}            \\
        \texttt{users}      & \texttt{Collection\textless User\textgreater}     & All users for whom the \texttt{WeekSchedule} is available     \\
        \texttt{days}       & \texttt{Collection\textless Weekday\textgreater}  & The \texttt{weekday}s belonging to the \texttt{WeekSchedule}           
    \end{tabular}}
    \caption{Table of fields in the \texttt{WeekSchedule} class.}
    \label{tbl:WeekSchedule}
\end{table}
\todo[inline]{Har gjort dem viewable nu, er ikke sikker på om overskriften skal være deri, det er jo captionens job? - Troels}

\noindent
The fields id, name and thumbnail are fairly straight forward and thus will not be discussed further.
These kind of simple fields exist for all the following classes, in general we will not go into detail with id, name and similar simple fields as their names combined with the short description is sufficient for them to be understood.
\begin{description}
    \item [lastEdit] \hfill \\ 
    The \texttt{lastEdit} field serves the same purpose as \texttt{lastEdit} described in \myref{pictogramendpoint}.
    \item [department] \hfill \\
    The \texttt{department} field is used to restrict access such that only users that are part of the department that owns the week schedule, can access that week schedule.
    \item [users] \hfill \\
    The \texttt{users} field defines what users the \texttt{WeekSchedule} is available to.
    The users in this collection pertains specifically to citizens as all guardians within a department can edit week schedules belonging to that department.
    This field allows for several citizens to use the same week schedule such that if the department goes on a trip, rather than having to create a week schedule for each individual citizen, they can use a shared one for shared activities.
    This is a feature not available in the live version of the system, yet it is something that the customers expressed a want for in our first meeting with them \citep{GIRAF20161stMeeting}.
    \item [days] \hfill \\
    The \texttt{days} field contains at most seven days which are represented by their own class, Weekday, described next.
    If a day is without activity it may be represented as either a Weekday, or lack thereof as an empty day does not require any information to create the view in the app.
\end{description}

\subsubsection{Weekday}
A \texttt{WeekSchedule} that has had any pictograms added to any related \texttt{Weekday}s, contains between one and seven weekdays.
These \texttt{} instances represent a single day of the week and contain information about the activities on a given day.
The table seen in \myref{tbl:Weekday} provides an overview and a short description of how we model a weekday in the REST API.

\begin{table}[ht]
\resizebox{\textwidth}{!}{%
    \centering
    \begin{tabular}{lll}
        \multicolumn{3}{c}{\Large{Class: \texttt{WeekSchedule}}}                                                                                \\
        \tblgrpsep
        \tblgrpsep
        \textbf{Field Name} & \textbf{Type}                                     & \textbf{Short Description}                                    \\
        \midrule
        \texttt{id}           & \texttt{long}                                   & Unique ID                                                     \\
        \texttt{day}          & \texttt{Enum\textless Day\textgreater}          & Defines which day of the week the \texttt{Weekday} represents \\
        \texttt{lastEdit}     & \texttt{Date}                                   & Timestamp for when the \texttt{Weekday} was last edited       \\
        \texttt{weekSchedule} & \texttt{WeekSchedule}                           & The \texttt{WeekSchedule} the \texttt{Weekday} belongs to     \\
        \texttt{frames}       & \texttt{List\textless WeekdayFrame\textgreater} & An ordered list of WeekdayFrames 
    \end{tabular}}
    \caption{Table of fields in the \texttt{Weekday} class.}
    \label{tbl:Weekday}
\end{table}
\todo[inline]{Har gjort dem viewable nu, er ikke sikker på om overskriften skal være deri, det er jo captionens job? - Troels}

\begin{description}
    \item [lastEdit] \hfill \\
    Having a \texttt{lastEdit} field on each \texttt{Weekday} as well as on each \texttt{WeekSchedule} allows for more specific information. 
    As such conflicts, as defined in \myref{ssec:policy}, are less likely to occur as the changes have to be targeted on the same Weekdays and not simply the \texttt{WeekSchedule}. 
    The lastEdit on \texttt{WeekSchedule} remains such that if no conflict occurs we will not have to check the \texttt{lastEdit} for each \texttt{Weekday} on a \texttt{WeekSchedule}.
    \item [weekSchedule] \hfill \\
    The \texttt{weekSchedule} field defines the \texttt{WeekSchedule} that a \texttt{Weekday} belongs to, a \texttt{Weekday} can only belong to one \texttt{WeekSchedule}.
    Allowing for \texttt{Weekday}s to be shared among \texttt{WeekSchedule}s would create more opportunities for the aforementioned conflicts to occur and in a more complicated manner, as such we choose not to allow this
    \item [frames] \hfill \\
    The \texttt{frames} field is a list used to contain \texttt{Frame}s on a \texttt{Weekday}.
    For this we use a class made specifically for linking \texttt{Frame}s and \texttt{Weekday}s together, \texttt{WeekdayFrame}.
    We use a list for this such that we can order the elements.
\end{description}

\subsubsection{WeekdayFrame}
In order to model that a pictogram has been added to day in the week schedule we use the \texttt{WeekdayFrame} class.
This class serves as the link between a \texttt{Weekday} and a \texttt{Frame}.
It is necessary that we have this intermediary link such that we can add additional information, in this case this information is ordering the \texttt{Frame}s in a list while allowing them to be reordered easily.
\myref{tbl:WeekdayFrame} shows what this class contains.

\begin{table}[ht]
\resizebox{\textwidth}{!}{%
    \centering
    \begin{tabular}{lll}
        \multicolumn{3}{c}{\Large{Class: \texttt{WeekSchedule}}}                                                                                        \\
        \tblgrpsep
        \tblgrpsep
        \textbf{Field Name} & \textbf{Type}                                   & \textbf{Short Description}                                              \\
        \midrule
        \texttt{id}         & \texttt{long}                                   & Unique ID                                                               \\
        \texttt{weekday}    & \texttt{Weekday}                                & The \texttt{Weekday} that a \texttt{Frame} should be added to           \\
        \texttt{frame}      & \texttt{Frame}                                  & The \texttt{Frame} that should be added to the \texttt{Weekday}         \\
        \texttt{index}      & \texttt{int}                                    & The order for which \texttt{Frame}s are stored                          \\
        \texttt{frames}     & \texttt{List\textless WeekdayFrame\textgreater} & The list that stores all \texttt{Frame}s contained in this frame
    \end{tabular}}
    \caption{Table that represents the \texttt{WeekdayFrame} class}
    \label{tbl:WeekdayFrame}
\end{table}

\begin{description}
    \item [weekday] \hfill \\
    This field contains the \texttt{Weekday} that the \texttt{Frame}(s) also contained in this class should be on.
    \item [frame] \hfill \\
    This field contains the specific \texttt{Frame} that should be added to the \texttt{Weekday}.
    \item [index] \hfill \\
    The \texttt{index} is used to order the list of \texttt{Frame}s contained for any class that has a list of \texttt{WeekdayFrame}s like this class and the \texttt{Weekday} class.
    \item [frames] \hfill \\
    As a \texttt{Frame} is used to model pictograms, choices and sequences, and sequences and choices consist of several pictograms, this must be modelled as well.
    This class fulfills that by allowing a \texttt{WeekdayFrame} to contain \texttt{Frame}s within \texttt{Frame}s.

\end{description}

\subsubsection{WeekdayFrameProgress}
As discussed earlier previously progress was not modelled in the system, but rather saved on a local file as the progress is specific to each citizen.
The \texttt{WeekdayFrameProgress} creates a link between a \texttt{User} and a \texttt{WeekdayFrame} such that progress can be tracked.

\begin{table}[ht]
\resizebox{\textwidth}{!}{%
    \centering
    \begin{tabular}{lll}
        \multicolumn{3}{c}{\Large{Class: \texttt{WeekSchedule}}}                                                                        \\
        \tblgrpsep
        \tblgrpsep
        \textbf{Field Name}   & \textbf{Type}                               & \textbf{Short Description}                                \\
        \midrule
        \texttt{id}           & \texttt{long}                               & Unique ID                                                 \\
        \texttt{user}         & \texttt{User}                               & The user for which progress is specified                  \\
        \texttt{weekdayFrame} & \texttt{WeekdayFrame}                       & The \texttt{WeekdayFrame} for which progress is specified \\
        \texttt{progress}     & \texttt{Enum\textless Progress\textgreater} & The Progress for the specified \texttt{WeekdayFrame}     
    \end{tabular}}
    \caption{Table that represents the \texttt{WeekdayFrameProgress} class}
    \label{tbl:WeekdayFrameProgress}
\end{table}
\todo[inline]{Lækkerfy this table Sass}

\begin{description}
    \item [user] \hfill \\
    The \texttt{User}, representing a citizen, for which progress must be tracked.
    \item [weekdayFrame] \hfill \\
    The \texttt{WeekdayFrame} for which a \texttt{User} must track their progress on
    \item [progress] \hfill \\
    The \texttt{Progress} enum contains four values, NotStarted, Done, Canceled and Active.
    This field is also nullable and will simply default to NotStarted if any attempt at information retrieval is made if the field has not been set yet.
    For tracking progress we specifically chose to not use booleans as the current model in the app does, as there is no scenario where any combination of values makes sense.
\end{description}

\subsection{Persistence}
As mentioned in \myref{sec:generalEP} the persistence layer contains the DAOs, SQL migrations and unittests.%*Salute* General EndPoint!
Furthermore for DAOs the rules for when to create a method for retrieving an object reduces it to a total of two methods for the entire Week Schedule model, both methods are used to retrieve \texttt{WeekSchedule}.
There are no methods for retrieving \texttt{WeekdayFrame}, \texttt{WeekdayFrameProgress} nor \texttt{Weekday}.
The first two are not relevant without the \texttt{Weekday} they refer to, and as we are working with objects by having the \texttt{WeekSchedule} we have all related \texttt{Weekday}s and thereby all relevant \texttt{WeekdayFrame}s and by extension \texttt{WeekdayFrameProgress}.

The two methods created for the DAO are \texttt{getAll(User user)} and \texttt{getById(long id)}.
While it may seem odd that the \texttt{getAll} method takes a user as parameter, this is to ensure that only available \texttt{WeekSchedule}s are retrieved, i.e. the user object is used to derive the department the user belongs to, such that only objects that the user should have access to are returned.
This may be used when a week schedule should be shared between several citizens, a list of all week schedules is required.
Similarly the \texttt{getById} method may then be useful when a specific week schedule is chosen, either while attempting to share a week schedule or simply when a citizen needs to access a week schedule.

These two methods may be the only in the \texttt{WeekScheduleDao}, meaning they are the only methods that directly retrieve a \texttt{WeekSchedule}, alas it is not the only way a \texttt{WeekSchedule} may be retrieved.
By using the \texttt{DepartmentDao} or the \texttt{UserDao} they can also be retrieved, as these both of these objects contain \texttt{WeekSchedule} objects.

\subsubsection{SQL}
\todo[inline]{I dont know what to put here, the general link between hibernate notations and tables would already be described in the two sections prior to this one}
\subsubsection{Unittests}
\todo[inline]{I dont know what to put here, the tests really arent interesting, nor do they affect the workings of the system}

\subsection{Work Remaining}
For this sprint only the prep work, i.e. the core and persistence layers, for the endpoint itself was finished.
The endpoint itself, that is to say the service layer, has yet to be made, and will be in the following sprint.
When we start creating the service layer it may reveal things that have been overlooked in the two other layers and subsequently cause changes.
\todo[inline]{I dont think there's enough content here to justify a subsec}