\section{REST API: Week Schedule}\label{sec:restws}

One of the focus points for this years GIRAF project have been the Week Schedule app.
The data for this application should be shared across devices, and some information across users.
Therefore it is crucial that the Week Schedule part of GIRAF is also part of its REST API.
This Section will explain the parts of the Week Schedule we developed during this sprint; the Core (model) and Persistence. 

\subsection{Model}
In this subsection we will introduce the model we made for the Week Schedule.
First we will explain what this is meant to be used for, that is the Week Schedule app, then we will explain our solution to the problem. 

In the Week Schedule app the view is split into seven views, one for each weekday. 
Each of these views contains an ordered list of a combination of pictograms, sequences and choices.
As explained earlier we model these as the \texttt{Frame}--superclass, from which they all inherit. % They inherit despite noone dies. 
Each frame in a week schedule additionally have a progress, which indicates the status of the frame. 
The progress' currently in use in the app are: Complete, Active, NotComplete and Canceled. 
The progress information is currently not stored on the server. \todo{This is correct right??? - T}
Since the progress information is personal, each user needs to separately have this information stored. 

\todo[inline]{WeekSchedule}

\todo[inline]{Weekday + Day enum}

\todo[inline]{WeekdayFrame}

\todo[inline]{WeekdayFrameProgress}

\subsection{Persistence}

\todo[inline]{SQL}

\todo[inline]{Unittests}

\subsection{Work to do} % Rename ... 

\todo[inline]{Endpoint}

