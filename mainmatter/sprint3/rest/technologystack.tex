% Core/Persistance/Service should be explained here. 

% The 3 modes should be explained (local only, local rest, remote db, and remote remote dev)

\section{Technologies and Tools}\label{sec:techstack}
\todo[inline]{Maybe this should be moved to appendix, such that we can refer to them when we talk about them. Only introduce the core few: Hibernate/Jackson/Spring ?}
In this section the technologies used in the GIRAF REST API and the development of it will be introduced.
In sprint 1 and 2, group SW615F16 worked on the REST API and they had decided on most of the technology stack to be used, however a few were introduced while we also are working on it. 
Below, we briefly introduce the technologies and tools used to develop and run the REST API:
\subsubsection{Technologies}
\begin{description}
    \item[Java] \hfill \\
        It was firstly decided that the REST API would be written in the Java programming language.
        This is most importantly to mirror the language used to develop the applications in GIRAF.
        Java is also a very popular language, it is number on 1 TIOBEs Index\footnote{\url{http://www.tiobe.com/tiobe_index}} as of writing this. 
        Java also has a big set of tools, frameworks etc. which makes it a good candidate for writing a REST API.

    \item[Hibernate] \hfill \\
        Hibernate\footnote{\url{http://hibernate.org/}} is a framework for object-relational mapping (ORM) which allows us to utilize high level objects while also storing them in a relational database. 
        We use Java annotations to configure it on a per variable and class base, this will be further explained later in \todo{ref her.}.

    \item[Jackson] \hfill \\
        Jackson\footnote{\url{http://wiki.fasterxml.com/JacksonHome}} is a Java library for processing JSON.
        We use it to (de)serialise objects between the client and the REST API. 

    \item[Spring] \hfill \\ 
        Spring\footnote{\url{https://spring.io/}} is a framework for dependency injection. 
        This allows us to decouple the configuration from the logic of the programs. 

    \item[RESTEasy] \hfill \\
        RESTEasy\footnote{\url{http://resteasy.jboss.org/}} is JBoss' (which also makes Wildfly) implementation of the JAX-RS 2.0 standard, which is a Java API used to make RESTful Web Services.
        RESTEasy provides annotations, according to the standard, which gives meaning to the methods in the service layer.
        Examples of these are: \texttt{@GET}, \texttt{@POST} and \texttt{@Produces}, these will be further explained when we use them in \todo[inline]{ref til brug her.}

    \item[MariaDB] \hfill \\ 
        MariaDB\footnote{\url{https://mariadb.org/}} is a drop in replacement for MySQL and was started as a fork of MySQL by its original developer when Oracle acquired MySQL.
        We use MariaDB on the database server as the back-end of the REST API.
        In it we will store all the persistent data. 
\end{description}
\subsubsection{Tools}
\begin{description}
    \item[Gradle] \hfill \\
        We use the Gradle Build system\footnote{\url{http://gradle.org/}} to automate our builds. 
        As with the rest of the GIRAF project Gradle is used for build automation. 
        We have previously explained what Gradle is in \myref{subsec:gradle}.

    \item[Wildfly] \hfill \\
        WildFly\footnote{\url{http://wildfly.org/}} is the application server which will run the REST API. 
        It can also be run on a development machine to test the API endpoints. 
        It is lightweight and has a fast startup speed, which is especially useful during development. 
        When it is run on a development machine it can either use a local or a remote database, such as the development database for the REST API. 

    \item[Enunciate] \hfill \\
        Enunciate\footnote{\url{https://github.com/stoicflame/enunciate}} is used to generate a web based documentation of the REST API endpoints automaticly. 
        This is useful for the people who will be interfacing with our REST API.
        The documentation is based on the code and the JavaDocs comments which are written in the code, thus this documentation is generated with minimal effort from the REST API developers.

    \item[JUnit] \hfill \\
        In order to verify the correctness of the code we unit test our code. 
        We have decided to use JUnit\footnote{\url{http://junit.org}} as our unit testing framework.
        JUnit is the de-facto standard for unit testing in Java\footnote{\url{http://www.methodsandtools.com/tools/tools.php?junit}}, and thusly is has good support within most IDEs and tools. 

    \item[H2 DB] \hfill \\
        The H2 Database Engine\footnote{\url{http://www.h2database.com/html/main.html}} is a low footprint local database written in Java. 
        We use it as a local database for testing, in its in-memory mode, which means that after each run all its data is deleted.
        It has a MySQL compatibility mode which means that it is reasonable to assume that any SQL executed onto it will do the same as if it was execute on MySql.

    \item[FlyWay] \hfill \\
        FlyWay\footnote{\url{https://flywaydb.org/}} is used to automatically apply the database migrations on start-up, both of H2 database during development and on the production server.
\end{description}
