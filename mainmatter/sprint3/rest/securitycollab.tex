\section{Security Meeting}\label{sec:securitycollab}
We held a meeting between ourselves, group SW615F16 and a representative from the security and PO group.
This section is developed in collaboration with group SW615F16 in order to document the purpose and results of this meeting.

The original motivation for making the REST API was enabling security and synchronization.
We have already documented how the previous solution was unable to achieve neither of these.
Up until now we had only considered security by authentication.
Since we moved the login from the Giraf application to the REST API, this have been improved greatly.

We had not considered the requirement that GIRAF should be usable while a device is offline.
To reconcile we decided to hold this meeting, such that we could design a proper solution which would follow all the requirements for a secure system to be used by a Danish municipality.

It is central that some information is limited to a single person.
However it is not a given that a single user will use the same tablet for every use of GIRAF while offline.
This poses a problem of giving users information for which they have no access, for use at later points in time by another user.
Access to another user's private information is in direct conflict with the security model proposed earlier by SW615F16, consisting only of authentication.

Some information is more private than other.
Some information is shared by all users of GIRAF, indiscriminate of which department they belong to.
Some information is shared by the users of a single department, but not across different departments.
Lastly, some information is private and only available to a single user.
We expect the great majority of information is either public or department specific, with a miniscule amount being private.

We define three access levels: One for all public information, one for all department shared information and one for private pr. user information.
We call these:
\begin{itemize}
    \item Public
    \item Protected
    \item Private
\end{itemize}

Under this model we can store the public and protected information on the tablet, and share it between all users while offline.
Since all the users are expected to be from the same department this poses no security issues.
We would still be unable to adequately secure private information, and as such would not provide access to it while offline.

To keep the private data from one person away from other people we delete private user data when the user logs out.
Likewise, when a guardian switches department, we delete all the department wide data.
Public data is never deleted, since everyone can access it at all times, anyway.

We expect the login process to follow a flow as seen on \myref{fig:loginworkflowrest}.
Our figure shows that it is possible to synchronise some data in the background, even before the user logs in.
Since a large number of pictograms, provided by Giraf, are stored as public we expect this to help with synchronisation speed on first login.

%\documentclass[crop,tikz]{standalone}
%\usetikzlibrary{shapes.multipart, arrows, matrix, automata, positioning, shadows, decorations.pathreplacing, calc}
\tikzstyle{wblock} = [rectangle, draw, fill=green!40, text width=6em, text badly centered, rounded corners, minimum height=3em]
\tikzstyle{memearrow} = [->, >=latex', shorten >=1pt, thick]
\tikzstyle{startstop} = [draw, ellipse,fill=red!20, minimum height=2em]
\tikzstyle{signs} = [wblock, fill=yellow!20, text width = 7.5em]
\begin{figure}[!ht]
    \centering
    \begin{tikzpicture}[node distance = 5em]
        \node [startstop] (start) {Open GIRAF};
        \node [wblock, below of = start] (selectdep) {Select Department};
        \node [wblock, below of = selectdep] (selectuser) {Select User};
        \node [wblock, below of = selectuser] (enterpassword) {Enter Password};
        \node [wblock, below of = enterpassword] (authed) {Authenticated};
        \node [wblock, below of = authed] (normaluse) {Pick User};
        \node [startstop, below of = normaluse] (givetouser) {Give to user};

        \draw [memearrow] (start) -- (selectdep);
        \draw [memearrow] (selectdep) -- (selectuser);
        \draw [memearrow] (selectuser) -- (enterpassword);
        \draw [memearrow] (enterpassword) -- (authed);
        \draw [memearrow] (authed) -- (normaluse);
        \draw [memearrow] (normaluse) -- (givetouser);

        % Right column
        \coordinate (startinvis) at ($(start)!0.5!(selectdep)$);
        \node [signs, right =7em of startinvis] (syncpublic) {Sync Public Data};
        \draw [memearrow,<->] (startinvis) -- (syncpublic);

        \coordinate (startinvis2) at ($(authed)!0.5!(normaluse)$);
        \node [signs, right =7em of startinvis2] (syncdep) {Sync Deparment Data};
        \draw [memearrow,<->] (startinvis2) -- (syncdep);

        \coordinate (startinvis3) at ($(normaluse)!0.5!(givetouser)$);
        \node [signs, right =7em of startinvis3] (syncuserdata) {Sync Private Data};
        \draw [memearrow,<->] (startinvis3) -- (syncuserdata);
    \end{tikzpicture}
    \caption{The suggested login workflow with the REST API.}
    \label{fig:loginworkflowrest}
\end{figure}
