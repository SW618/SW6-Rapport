\section{Security Meeting}\label{sec:securitycollab}
We held a meeting between ourselves, group SW615F16 and a representative from the security and PO group regarding security and synchronisation for the REST API.
We had to find a way to be in accordance with the Act on the Processing of Personal Data \§ 41.3.\footnote{\url{https://www.retsinformation.dk/forms/r0710.aspx?id=828}}.
Which states that personal data should be stored securely, inaccessible to third parties.
This section is developed in collaboration with group SW615F16 in order to document the purpose and results of this meeting.

The original motivation for making the REST API was enabling security and synchronization.
We have already documented how the previous solution was unable to achieve neither of these.
Up until now we had only considered security by authentication.
Since we moved the login from the GIRAF application to the REST API, this have been improved greatly.

We had not considered the requirement that GIRAF should be usable while a device is offline.
To reconcile we decided to hold this meeting, such that we could design a proper solution which would follow all the requirements for a secure system to be used by a Danish municipality.

It is central that some information is limited to a single person.
However it is not a given that a single user will use the same tablet for every use of GIRAF while offline.
When the citizens arrive at an institution they simply get a random tablet from a bundle, and as such it is not certain that they will get the same one each day.
This poses a problem of giving users information for which they should not have access, for use at later points in time by another user. 
Accessing another user's private information is therefore not solved by just having authentication; encryption is also needed if private information is to be located on the tablets.

Some information is more private than other. 
Some information is shared by all users of GIRAF, indiscriminate of which department they belong to.
Some information is shared by the users of a single department, but not across different departments.
Lastly, some information is private and only available to a single user.
We expect the great majority of information is either public or department specific, with a minuscule amount being private.

We define three access levels: One for all public information, one for all department shared information and one for private per user information.
We call these:
\begin{itemize}
    \item Public
    \item Protected
    \item Private
\end{itemize}

Under this model we can store the public and protected information on the tablet, and share it between all users of the department while offline.
Since all the users are expected to be from the same department this poses no security issues, as multiple departments is not the intended use case of GIRAF.
This model does not account for private data as this should only be available for the user it belongs to, and thus the data cannot safely be stored on a tablet as these are shared.
Therefore to keep the private data of a citizen away from another citizen, we delete private user data when the user logs out.
Likewise, when a guardian switches the department of a tablet, we delete all data of the old department on the tablet.
Public data is never deleted, since everyone can access it at all times, anyway.

In order to secure the data on a tablet in accordance with the Act on Processing of Personal Data, we ensure that only authorised users gain access to data, and that the data is both stored and transmitted in encrypted form.


We expect the initial set--up process of a Guardian for GIRAF to follow a flow as seen on \myref{fig:loginworkflowrest}.
When GIRAF is opened it is possible to synchronise some data in the background, even before the user logs in.
Since a large number of pictograms, provided by GIRAF, are stored as public, we expect this to be the largest amount of data to be synchronised at any time.
Thus the earlier this process is started, the faster the initial set--up will be.
The Guardian then selects a Department, which will allow them to pick a Guardian from the chosen department to login as.
They are then prompted for a password, which will if successful start a download of all the data of the department they chose.
After this step they pick a citizen which will use the tablet, which will then start the download of personal data.
After the initial set--up, the \textit{Select Department} step is skipped, and the first step is instead \texit{Select Guardian}.

%\documentclass[crop,tikz]{standalone}
%\usetikzlibrary{shapes.multipart, arrows, matrix, automata, positioning, shadows, decorations.pathreplacing, calc}
\tikzstyle{wblock} = [rectangle, draw, fill=green!40, text width=6em, text badly centered, rounded corners, minimum height=3em]
\tikzstyle{memearrow} = [->, >=latex', shorten >=1pt, thick]
\tikzstyle{startstop} = [draw, ellipse,fill=red!20, minimum height=2em]
\tikzstyle{signs} = [wblock, fill=yellow!20, text width = 7.5em]
\begin{figure}[!ht]
    \centering
    \begin{tikzpicture}[node distance = 5em]
        \node [startstop] (start) {Open GIRAF};
        \node [wblock, below of = start] (selectdep) {Select Department};
        \node [wblock, below of = selectdep] (selectuser) {Select Guardian};
        \node [wblock, below of = selectuser] (enterpassword) {Enter Password};
        \node [wblock, below of = enterpassword] (authed) {Authenticated};
        \node [wblock, below of = authed] (normaluse) {Pick user};
        \node [startstop, below of = normaluse] (givetouser) {Give to user};

        \draw [memearrow] (start) -- (selectdep);
        \draw [memearrow] (selectdep) -- (selectuser);
        \draw [memearrow] (selectuser) -- (enterpassword);
        \draw [memearrow] (enterpassword) -- (authed);
        \draw [memearrow] (authed) -- (normaluse);
        \draw [memearrow] (normaluse) -- (givetouser);

        % Right column
        \coordinate (startinvis) at ($(start)!0.5!(selectdep)$);
        \node [signs, right =7em of startinvis] (syncpublic) {Sync Public Data};
        \draw [memearrow,<->] (startinvis) -- (syncpublic);

        \coordinate (startinvis2) at ($(authed)!0.5!(normaluse)$);
        \node [signs, right =7em of startinvis2] (syncdep) {Sync Deparment Data};
        \draw [memearrow,<->] (startinvis2) -- (syncdep);

        \coordinate (startinvis3) at ($(normaluse)!0.5!(givetouser)$);
        \node [signs, right =7em of startinvis3] (syncuserdata) {Sync Private Data};
        \draw [memearrow,<->] (startinvis3) -- (syncuserdata);
    \end{tikzpicture}
    \caption{The suggested initial set--up of GIRAF with the REST API.}
    \label{fig:loginworkflowrest}
\end{figure}
