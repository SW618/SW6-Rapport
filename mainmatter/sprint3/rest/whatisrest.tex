\section{What Is REST}
REST is an architectural style for distributed hypermedia systems.
REST is also defined as a hybrid style due to its constraints being derived from several other network-based architectural styles.
The idea is that rather than using complex protocols such as Remote Procedure Calls (\textbf{RPC}) to connect between entities, the simpler HTTP requests are used instead.
Originally defined in 2000 by Roy Fielding in his doctoral dissertation, \citep{fielding2000rest}, REST establishes a set of constraints that positively impacts the following set of non-functional properties. 
\begin{multicols}{3}
\begin{itemize}
    \item Performance
    \item Scalability
    \item Simplicity
    \item Visibility
    \item Portability
    \item Reliability
\end{itemize}      
\end{multicols}
An application that adheres to the constraints of REST is considered to be RESTful.
Next we will present each of the constraints and how they relate to the aforementioned non-functional properties.

\begin{description}
    \item [Client-Server\label{client-server-rest}] The first constraint we will discuss is derived from the hierarchical client-server architechture.
    Separation of concern is the underlying concept for the client-server architecture, in essence the client serves as a trigger process, and the server as a reactive process.
    As the reactive process, the server is more often than not a non-terminating process waiting for a request from a trigger.
    When one of the multiple clients requests data, the client cares not for how the data is acquired, nor for how it is stored, as long as it is received in the agreed upon format.
    Likewise the server cares not for how the client represents the data, nor for which state the client is in as long as requests are made in the agreed upon format.
    This separation of concern affects three of the aforementioned non-functional properties.
    \begin{itemize}
        \item Simplicity: By not having to consider client specific data, components are easier to represent.
        \item Scalability: By simplifying the server components, the server components becomes scalable by allowing them to evolve independantly.
        \item Portability: By removing concerns between server and client specific details, significantly improving the portability of client side code, as only the language between the two is important.
    \end{itemize}\todo[inline]{Tikz Figure of client server architechture?}
    %Scalability, Simplicity, Portability 
    \item [Stateless] Adding to the constraints we further require that communication between client and server must be stateless.
    In making communication stateless, all requests must contain every bit of information required to perform the request, i.e. no context stored on the server can be used.
    As with most decisions this also has its downside, with no context available the overhead for each request may be higher for repetitive requests.
    Making the communication stateless does however improve visibility, reliability and scalability.
    \begin{itemize}
        \item Visibility: Visibility is increased as there is no need to look beyond the request in order to discern the purpose of the request contrary to RCP.
        \item Reliability: Reliability is increased as partial failures are more recoverable by completely removing the chance for invalid states.
        \item Scalability: Scalability is improved by not having to store information regarding temporary states.
    \end{itemize}
    %Performance, Scalability, Simplicity, Visibility, Reliability
    \item [Cacheable] In order to increase efficiency the constraint of cacheable data is added.
    The constraint requires that when a response to a request specifies whether the data is cacheable or non-cacheable which may remove the need for computing a later request.
    When a response is noted as cacheable, a timeout is also required which the client must adhere to.
    This constraint may slightly reduce reliability if the data in the cache becomes outdated before timeout, however for GIRAF this should only rarely if ever be an issue as two users working in parallel will rarely affect each other and in that case it is only at the point of synchronising.
    Other non-functional properties that are instead positively affected are scalability and performance.
    \begin{itemize}
        \item Scalability: By clients not having to always request, the server can handlre more requests from more clients within a given time frame.
        \item Performance: With some requests not needing a results which is computed by the server but can be partially or completely handled through the cache, average latency is improved.
    \end{itemize}
    %Performance, Scalability, Reliability
    \item [Uniform Interface] The fourth constraint Uniform Interface is expressed by Fielding as \enquote{The central feature that distinguishes the REST architectural style from other network-based styles}\citep{fielding2000rest}.
    Applying the principle of generality to the interface between server and client achieves low coupling between components allowing components to evolve independently, as long as the interface persists. 
    The interface itself follows four main principles.
    Firstly it uses resource identifiers to request resources from the server, the resources themselves are represented differently in requests than the way they are stored, e.g. for GIRAF we represent our data using the JSON format, but the data is not stored in this format.
    Secondly clients must be able to manipulate resources through representations, this means that when a client has a representation of a resource, the representation contains enough data such that the client can and may alter or delete the resource, given the client is allowed to do so.
    Thirdly messages must be self-descriptive such that a process can be executed as intended without further inquiry.
    The last principle is Hypermedia as the Engine of Application State (HATEOAS), simply put interaction by the client with a network application is through hypermedia provided by the server.
    The client needs no knowledge of available actions on any given data, as the server will only provide hypermedia for actions that are available in the state the resource is in.

    As with all generalisation efficiency is affected as one cannot tailor the interface to a systems unique needs.
    This generalisation is however not without its perks, both simplicity and visibility is also affected.
    \begin{itemize}
        \item Simplicity: By adhering to generalisation the overall structure of the system architecture is simplified.
        \item Visibility: Through simplified architecture the visibility of interactions is also increased.
    \end{itemize}
    %Simplicity, Visibility
    \item [Layered System] The fifth constraint of having a layered system adds hierarchical layers to the architecture, encapsulates the behavior of a layer to itself and its immediately used layers.
    By layering the system we acquire low coupling while also keeping the overall complexity of the system low.
    Layers can serve a variety of purposes from protecting legacy services to simplifying components by isolating less used services.
    Combined with the system being cacheable, shared caches can also be used to gain positive effect.
    \begin{itemize}
        \item Performance: While the layers themselves increase overhead and thus increasing latency, shared caches can be used to counter this by reducing latency.
        \item Scalability: Intermediary layers can be used for load balancing thus improving scalability.
        \item Simplicity: Layers can be used to simplify components by separating less frequently used functionality from often used functionality, and also by encapsulating certain functionality in layers such that a logical separation of functionality is achieved.
    \end{itemize} 
    %Performance, Scalability, Simplicity
    \item [Code-On-Demand] is the last constraint and is also the only constraint that can be considered optional.
    Since this constraint will not be used for this project, it will not be further described here, for more information on Code-On-Demand we refer to Fieldings doctoral dissertation \citep{fielding2000rest}.
\end{description}

These constraints is what will make an API RESTful, which means that the API for GIRAF will also adhere to these constraints.
The next section will present the design choices 